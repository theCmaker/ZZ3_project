Nous présentons dans cette partie un ensemble de tests que nous avons effectués pour contrôler le bon fonctionnement de nos algorithmes. 

Nous avons réalisé des graphiques afin d'avoir une meilleure interprétation des résultats obtenus.

Sur ces graphiques, nous avons choisi de représenter les mobiles $M_i$ non-interceptés par des croix vertes (\tikz[baseline=-0.5ex]{\node[mobile,inner sep=0,outer sep=0]{\mobile};}), et les mobiles $M_i$ interceptés par des croix rouges (\tikz[baseline=-0.5ex]{\node[caught,inner sep=0,outer sep=0]{\mobile};}).

Les vecteurs vitesse et la trajectoire empruntée par les mobiles sont indiqués en vert par des vecteurs (\tikz[baseline=-0.5ex]{\draw[speed] (0,0) -- (1,0);}) et des lignes pointillées (\tikz[baseline=-0.5ex]{\draw[camino] (0,0) -- (1,0);}).

La position initiale de l'intercepteur est repérée par un carré bleu (\tikz[baseline=-0.5ex]{\node[interceptor,inner sep=0,outer sep=0]{\interceptor};}) et ses positions successives par des croix bleues (\tikz[baseline=-0.5ex]{\node[interceptor,inner sep=0,outer sep=0]{\mobile};}). La date de la dernière interception est indiquée au-dessus de la position où elle a lieu.

\section{Test \no1: Tous les mobiles interceptés, séquences différentes}
  \begin{listing}[H]
    \textfile{../tests/test_1/test_1.data}
    \caption{test\_1.data}
  \end{listing}

  \begin{figure}[H]
    \begin{center}
      \boxed{
      \begin{tikzpicture}[scale=0.7]
        \draw[grided,step=1.0,thin] (-2.000000,-8.000000) grid (14.363306,9.297633);
\draw (-2.000000,0) -- coordinate (x axis mid) (14.363306,0);
\draw (0,-8.000000) -- coordinate (y axis mid) (0,9.297633);
\foreach \x in {-2,...,14}
  \draw (\x,1pt) -- (\x,-3pt) node[anchor=north] {\x};
\foreach \y in {-8,...,9}
  \draw (1pt,\y) -- (-3pt,\y) node[anchor=east] {\y};
\node[interceptor] (I0) at (0.000000,0.000000) {\interceptor};
\node[mobile,anchor=center] (M0) at (2.000000,2.000000) {\mobile};
\node[mobile] at (M0.south east) {$M_0$};
\draw[speed] (M0.center) -- ($ (M0.center) + (-1.000000,0.000000) $);
\node[mobile,anchor=center] (M1) at (-1.000000,-1.000000) {\mobile};
\node[mobile] at (M1.south east) {$M_1$};
\draw[speed] (M1.center) -- ($ (M1.center) + (0.000000,2.000000) $);
\node[mobile,anchor=center] (M2) at (4.000000,5.000000) {\mobile};
\node[mobile] at (M2.south east) {$M_2$};
\draw[speed] (M2.center) -- ($ (M2.center) + (-1.000000,-0.300000) $);
\node[mobile,anchor=center] (M3) at (2.000000,4.000000) {\mobile};
\node[mobile] at (M3.south east) {$M_3$};
\draw[speed] (M3.center) -- ($ (M3.center) + (0.750000,-0.690000) $);
\draw[camino] (M0) -- (0.902832,2.000000);
\draw[interceptor] (0.000000,0.000000) -- (0.902832,2.000000);
\node[interceptor] at (0.902832,2.000000) {\mobile};
\node[caught] at (M0) {\mobile};
\draw[camino] (M2) -- (1.685421,4.305626);
\draw[interceptor] (0.902832,2.000000) -- (1.685421,4.305626);
\node[interceptor] at (1.685421,4.305626) {\mobile};
\node[caught] at (M2) {\mobile};
\draw[camino] (M1) -- (-1.000000,9.297633);
\draw[interceptor] (1.685421,4.305626) -- (-1.000000,9.297633);
\node[interceptor] at (-1.000000,9.297633) {\mobile};
\node[caught] at (M1) {\mobile};
\draw[camino] (M3) -- (14.363306,-7.374242);
\draw[interceptor] (-1.000000,9.297633) -- (14.363306,-7.374242);
\node[interceptor] at (14.363306,-7.374242) {\mobile};
\node[caught] at (M3) {\mobile};
\draw[interceptor] (14.363306,-7.374242) node[anchor=south east] {$t=16.484 \text{ u.t.}$};

      \end{tikzpicture}}
    \end{center}
    \caption{Heuristique $H_0$: Test \no1}
    \label{fig:H0_1}
  \end{figure}

  \begin{table}[H]
    \centering
    \begin{tabular}{|c|c|c|}
  \hline\textbf{\No mobile} & \textbf{Position interception} & \textbf{Date interception (u.t.)} \\ \hline 
  0	& $\coord{0.461}{2.000}$	 & $1.0262$ \\ \hline
  1	& $\coord{-1.000}{2.652}$	 & $1.8260$ \\ \hline
  2	& $\coord{0.959}{4.088}$	 & $3.0406$ \\ \hline
  3	& $\coord{5.797}{-0.367}$	 & $6.3288$ \\ \hline
\end{tabular}

    \caption{Heuristique $H_0$: Résultats test \no1}
    \label{tab:H0_1}
  \end{table}

  \begin{figure}[H]
    \begin{center}
      \boxed{
      \begin{tikzpicture}[scale=0.7]
        \draw[grided,step=1.0,thin] (-13.000000,-9.000000) grid (8.360891,9.000000);
\draw (-13.000000,0) -- coordinate (x axis mid) (8.360891,0);
\draw (0,-9.000000) -- coordinate (y axis mid) (0,9.000000);
\foreach \x in {-13,...,8}
  \draw (\x,1pt) -- (\x,-3pt) node[anchor=north] {\x};
\foreach \y in {-9,...,9}
  \draw (1pt,\y) -- (-3pt,\y) node[anchor=east] {\y};
\node[interceptor] (I0) at (2.000000,9.000000) {\interceptor};
\node[mobile,anchor=center] (M0) at (7.140297,8.876400) {\mobile};
\node[mobile] at (M0.south east) {$M_0$};
\node[mobile,anchor=center] (M1) at (4.767767,5.291864) {\mobile};
\node[mobile] at (M1.south east) {$M_1$};
\draw[speed] (M1.center) -- ($ (M1.center) + (-0.811979,-0.290230) $);
\node[mobile,anchor=center] (M2) at (-7.277143,-4.215198) {\mobile};
\node[mobile] at (M2.south east) {$M_2$};
\draw[speed] (M2.center) -- ($ (M2.center) + (0.458222,0.901331) $);
\node[mobile,anchor=center] (M3) at (-9.317039,-8.399123) {\mobile};
\node[mobile] at (M3.south east) {$M_3$};
\draw[speed] (M3.center) -- ($ (M3.center) + (-0.209148,0.404874) $);
\node[mobile,anchor=center] (M4) at (8.360891,5.869070) {\mobile};
\node[mobile] at (M4.south east) {$M_4$};
\draw[speed] (M4.center) -- ($ (M4.center) + (-0.505128,0.038969) $);
\node[mobile,anchor=center] (M5) at (2.371453,-8.866265) {\mobile};
\node[mobile] at (M5.south east) {$M_5$};
\draw[speed] (M5.center) -- ($ (M5.center) + (-0.225212,0.200970) $);
\draw[interceptor] (2.000000,9.000000) -- (7.140297,8.876400);
\node[interceptor] at (7.140297,8.876400) {\mobile};
\node[caught] at (M0) {\mobile};
\draw[camino] (M1) -- (-1.586399,3.020660);
\draw[interceptor] (7.140297,8.876400) -- (-1.586399,3.020660);
\node[interceptor] at (-1.586399,3.020660) {\mobile};
\node[caught] at (M1) {\mobile};
\draw[camino] (M2) -- (-3.277717,3.651746);
\draw[interceptor] (-1.586399,3.020660) -- (-3.277717,3.651746);
\node[interceptor] at (-3.277717,3.651746) {\mobile};
\node[caught] at (M2) {\mobile};
\draw[camino] (M3) -- (-12.292086,-2.639952);
\draw[interceptor] (-3.277717,3.651746) -- (-12.292086,-2.639952);
\node[interceptor] at (-12.292086,-2.639952) {\mobile};
\node[caught] at (M3) {\mobile};
\draw[camino] (M4) -- (-2.280671,6.690032);
\draw[interceptor] (-12.292086,-2.639952) -- (-2.280671,6.690032);
\node[interceptor] at (-2.280671,6.690032) {\mobile};
\node[caught] at (M4) {\mobile};
\draw[camino] (M5) -- (-3.539516,-3.591557);
\draw[interceptor] (-2.280671,6.690032) -- (-3.539516,-3.591557);
\node[interceptor] at (-3.539516,-3.591557) {\mobile};
\node[caught] at (M5) {\mobile};
\draw[interceptor] (-3.539516,-3.591557) node[anchor=south east] {$t=26.246 \text{ u.t.}$};

      \end{tikzpicture}}
    \end{center}
    \caption{Heuristique $H_1$: Test \no1}
    \label{fig:H1_1}
  \end{figure}

  \begin{table}[H]
    \centering
    \begin{tabular}{|c|c|c|}
  \hline\textbf{\No mobile} & \textbf{Position interception} & \textbf{Date interception (u.t.)} \\ \hline 
  0	& $\coord{0.000}{2.000}$	 & $1.0000$ \\ \hline
  1	& $\coord{0.000}{2.000}$	 & $1.0000$ \\ \hline
  2	& $\coord{7.323}{6.597}$	 & $5.3233$ \\ \hline
  3	& $\coord{7.249}{4.249}$	 & $6.4979$ \\ \hline
  4	& $\coord{-8.089}{5.089}$	 & $14.1785$ \\ \hline
\end{tabular}

    \caption{Heuristique $H_1$: Résultats test \no1}
    \label{tab:H1_1}
  \end{table}

  \begin{figure}[H]
    \centering
    \boxed{
    \begin{tikzpicture}[yscale=0.5]
      \draw[grided,step=1.0,thin] (0,0) grid (5,13.000000);
\draw (0,0) -- coordinate (x axis mid) (5,0);
\draw (0,0) -- coordinate (y axis mid) (0,13.000000);
\foreach \x in {0,...,5}
  \draw (\x,1pt) -- (\x,-3pt) node[anchor=north] {\x};
\node[h0] at (1,7.500000) {$H_0$};
\node[h1] at (1,5.500000) {$H_1$};
\foreach \y in {0,1,...,13}
  \draw (1pt,\y) -- (-3pt,\y) node[anchor=east] {\y};
\draw (2.500000,-2) node[anchor=north] {Nombre de mobiles interceptés};
\draw (-0.75,6.500000) node[rotate=90,anchor=south] {Temps nécessaire (u.t)};
\node[h0] at (0,0) {\cross};
\draw[h0] (0,0.000000) -- (1,1.000000);
\node[h0] at(1,1.000000) {\cross};
\draw[h0] (1,1.000000) -- (2,1.000000);
\node[h0] at(2,1.000000) {\cross};
\draw[h0] (2,1.000000) -- (3,2.769642);
\node[h0] at(3,2.769642) {\cross};
\draw[h0] (3,2.769642) -- (4,9.018394);
\node[h0] at(4,9.018394) {\cross};
\draw[h0] (4,9.018394) -- (5,12.344612);
\node[h0] at(5,12.344612) {\cross};
\draw[grided,step=1.0,thin] (0,0) grid (5,15.000000);
\draw (0,0) -- coordinate (x axis mid) (5,0);
\draw (0,0) -- coordinate (y axis mid) (0,15.000000);
\node[h1] at (0,0) {\cross};
\draw[h1] (0,0.000000) -- (1,1.000000);
\node[h1] at(1,1.000000) {\cross};
\draw[h1] (1,1.000000) -- (2,1.000000);
\node[h1] at(2,1.000000) {\cross};
\draw[h1] (2,1.000000) -- (3,5.323262);
\node[h1] at(3,5.323262) {\cross};
\draw[h1] (3,5.323262) -- (4,6.497872);
\node[h1] at(4,6.497872) {\cross};
\draw[h1] (4,6.497872) -- (5,14.178453);
\node[h1] at(5,14.178453) {\cross};

    \end{tikzpicture}}
    \caption{Comparaison de $H_0$ et de $H_1$: test \no1}
    \label{fig:comp_1}
  \end{figure}

\section{Test \no2: Résultats identiques et mobile non-intercepté}
  \begin{listing}[H]
    \textfile{../tests/test_2/test_2.data}
    \caption{test\_2.data}
  \end{listing}

  \begin{figure}[H]
    \begin{center}
      \boxed{
      \begin{tikzpicture}[scale=1]
        \draw[grided,step=1.0,thin] (-2.000000,-8.000000) grid (14.363306,9.297633);
\draw (-2.000000,0) -- coordinate (x axis mid) (14.363306,0);
\draw (0,-8.000000) -- coordinate (y axis mid) (0,9.297633);
\foreach \x in {-2,...,14}
  \draw (\x,1pt) -- (\x,-3pt) node[anchor=north] {\x};
\foreach \y in {-8,...,9}
  \draw (1pt,\y) -- (-3pt,\y) node[anchor=east] {\y};
\node[interceptor] (I0) at (0.000000,0.000000) {\interceptor};
\node[mobile,anchor=center] (M0) at (2.000000,2.000000) {\mobile};
\node[mobile] at (M0.south east) {$M_0$};
\draw[speed] (M0.center) -- ($ (M0.center) + (-1.000000,0.000000) $);
\node[mobile,anchor=center] (M1) at (-1.000000,-1.000000) {\mobile};
\node[mobile] at (M1.south east) {$M_1$};
\draw[speed] (M1.center) -- ($ (M1.center) + (0.000000,2.000000) $);
\node[mobile,anchor=center] (M2) at (4.000000,5.000000) {\mobile};
\node[mobile] at (M2.south east) {$M_2$};
\draw[speed] (M2.center) -- ($ (M2.center) + (-1.000000,-0.300000) $);
\node[mobile,anchor=center] (M3) at (2.000000,4.000000) {\mobile};
\node[mobile] at (M3.south east) {$M_3$};
\draw[speed] (M3.center) -- ($ (M3.center) + (0.750000,-0.690000) $);
\draw[camino] (M0) -- (0.902832,2.000000);
\draw[interceptor] (0.000000,0.000000) -- (0.902832,2.000000);
\node[interceptor] at (0.902832,2.000000) {\mobile};
\node[caught] at (M0) {\mobile};
\draw[camino] (M2) -- (1.685421,4.305626);
\draw[interceptor] (0.902832,2.000000) -- (1.685421,4.305626);
\node[interceptor] at (1.685421,4.305626) {\mobile};
\node[caught] at (M2) {\mobile};
\draw[camino] (M1) -- (-1.000000,9.297633);
\draw[interceptor] (1.685421,4.305626) -- (-1.000000,9.297633);
\node[interceptor] at (-1.000000,9.297633) {\mobile};
\node[caught] at (M1) {\mobile};
\draw[camino] (M3) -- (14.363306,-7.374242);
\draw[interceptor] (-1.000000,9.297633) -- (14.363306,-7.374242);
\node[interceptor] at (14.363306,-7.374242) {\mobile};
\node[caught] at (M3) {\mobile};
\draw[interceptor] (14.363306,-7.374242) node[anchor=south east] {$t=16.484 \text{ u.t.}$};

      \end{tikzpicture}}
    \end{center}
    \caption{Heuristique $H_0$: Test \no2}
    \label{fig:H0_2}
  \end{figure}

  \begin{table}[H]
    \centering
    \begin{tabular}{|c|c|c|}
  \hline\textbf{\No mobile} & \textbf{Position interception} & \textbf{Date interception (u.t.)} \\ \hline 
  0	& $\coord{0.461}{2.000}$	 & $1.0262$ \\ \hline
  1	& $\coord{-1.000}{2.652}$	 & $1.8260$ \\ \hline
  2	& $\coord{0.959}{4.088}$	 & $3.0406$ \\ \hline
  3	& $\coord{5.797}{-0.367}$	 & $6.3288$ \\ \hline
\end{tabular}

    \caption{Heuristique $H_0$: Résultats test \no2}
    \label{tab:H0_2}
  \end{table}

  \begin{figure}[H]
    \begin{center}
      \boxed{
      \begin{tikzpicture}[scale=1]
        \draw[grided,step=1.0,thin] (-13.000000,-9.000000) grid (8.360891,9.000000);
\draw (-13.000000,0) -- coordinate (x axis mid) (8.360891,0);
\draw (0,-9.000000) -- coordinate (y axis mid) (0,9.000000);
\foreach \x in {-13,...,8}
  \draw (\x,1pt) -- (\x,-3pt) node[anchor=north] {\x};
\foreach \y in {-9,...,9}
  \draw (1pt,\y) -- (-3pt,\y) node[anchor=east] {\y};
\node[interceptor] (I0) at (2.000000,9.000000) {\interceptor};
\node[mobile,anchor=center] (M0) at (7.140297,8.876400) {\mobile};
\node[mobile] at (M0.south east) {$M_0$};
\node[mobile,anchor=center] (M1) at (4.767767,5.291864) {\mobile};
\node[mobile] at (M1.south east) {$M_1$};
\draw[speed] (M1.center) -- ($ (M1.center) + (-0.811979,-0.290230) $);
\node[mobile,anchor=center] (M2) at (-7.277143,-4.215198) {\mobile};
\node[mobile] at (M2.south east) {$M_2$};
\draw[speed] (M2.center) -- ($ (M2.center) + (0.458222,0.901331) $);
\node[mobile,anchor=center] (M3) at (-9.317039,-8.399123) {\mobile};
\node[mobile] at (M3.south east) {$M_3$};
\draw[speed] (M3.center) -- ($ (M3.center) + (-0.209148,0.404874) $);
\node[mobile,anchor=center] (M4) at (8.360891,5.869070) {\mobile};
\node[mobile] at (M4.south east) {$M_4$};
\draw[speed] (M4.center) -- ($ (M4.center) + (-0.505128,0.038969) $);
\node[mobile,anchor=center] (M5) at (2.371453,-8.866265) {\mobile};
\node[mobile] at (M5.south east) {$M_5$};
\draw[speed] (M5.center) -- ($ (M5.center) + (-0.225212,0.200970) $);
\draw[interceptor] (2.000000,9.000000) -- (7.140297,8.876400);
\node[interceptor] at (7.140297,8.876400) {\mobile};
\node[caught] at (M0) {\mobile};
\draw[camino] (M1) -- (-1.586399,3.020660);
\draw[interceptor] (7.140297,8.876400) -- (-1.586399,3.020660);
\node[interceptor] at (-1.586399,3.020660) {\mobile};
\node[caught] at (M1) {\mobile};
\draw[camino] (M2) -- (-3.277717,3.651746);
\draw[interceptor] (-1.586399,3.020660) -- (-3.277717,3.651746);
\node[interceptor] at (-3.277717,3.651746) {\mobile};
\node[caught] at (M2) {\mobile};
\draw[camino] (M3) -- (-12.292086,-2.639952);
\draw[interceptor] (-3.277717,3.651746) -- (-12.292086,-2.639952);
\node[interceptor] at (-12.292086,-2.639952) {\mobile};
\node[caught] at (M3) {\mobile};
\draw[camino] (M4) -- (-2.280671,6.690032);
\draw[interceptor] (-12.292086,-2.639952) -- (-2.280671,6.690032);
\node[interceptor] at (-2.280671,6.690032) {\mobile};
\node[caught] at (M4) {\mobile};
\draw[camino] (M5) -- (-3.539516,-3.591557);
\draw[interceptor] (-2.280671,6.690032) -- (-3.539516,-3.591557);
\node[interceptor] at (-3.539516,-3.591557) {\mobile};
\node[caught] at (M5) {\mobile};
\draw[interceptor] (-3.539516,-3.591557) node[anchor=south east] {$t=26.246 \text{ u.t.}$};

      \end{tikzpicture}}
    \end{center}
    \caption{Heuristique $H_1$: Test \no2}
    \label{fig:H1_2}
  \end{figure}

  \begin{table}[H]
    \centering
    \begin{tabular}{|c|c|c|}
  \hline\textbf{\No mobile} & \textbf{Position interception} & \textbf{Date interception (u.t.)} \\ \hline 
  0	& $\coord{0.000}{2.000}$	 & $1.0000$ \\ \hline
  1	& $\coord{0.000}{2.000}$	 & $1.0000$ \\ \hline
  2	& $\coord{7.323}{6.597}$	 & $5.3233$ \\ \hline
  3	& $\coord{7.249}{4.249}$	 & $6.4979$ \\ \hline
  4	& $\coord{-8.089}{5.089}$	 & $14.1785$ \\ \hline
\end{tabular}

    \caption{Heuristique $H_1$: Résultats test \no2}
    \label{tab:H1_2}
  \end{table}

  \begin{figure}[H]
    \centering
    \boxed{
    \begin{tikzpicture}[yscale=0.5]
      \draw[grided,step=1.0,thin] (0,0) grid (5,13.000000);
\draw (0,0) -- coordinate (x axis mid) (5,0);
\draw (0,0) -- coordinate (y axis mid) (0,13.000000);
\foreach \x in {0,...,5}
  \draw (\x,1pt) -- (\x,-3pt) node[anchor=north] {\x};
\node[h0] at (1,7.500000) {$H_0$};
\node[h1] at (1,5.500000) {$H_1$};
\foreach \y in {0,1,...,13}
  \draw (1pt,\y) -- (-3pt,\y) node[anchor=east] {\y};
\draw (2.500000,-2) node[anchor=north] {Nombre de mobiles interceptés};
\draw (-0.75,6.500000) node[rotate=90,anchor=south] {Temps nécessaire (u.t)};
\node[h0] at (0,0) {\cross};
\draw[h0] (0,0.000000) -- (1,1.000000);
\node[h0] at(1,1.000000) {\cross};
\draw[h0] (1,1.000000) -- (2,1.000000);
\node[h0] at(2,1.000000) {\cross};
\draw[h0] (2,1.000000) -- (3,2.769642);
\node[h0] at(3,2.769642) {\cross};
\draw[h0] (3,2.769642) -- (4,9.018394);
\node[h0] at(4,9.018394) {\cross};
\draw[h0] (4,9.018394) -- (5,12.344612);
\node[h0] at(5,12.344612) {\cross};
\draw[grided,step=1.0,thin] (0,0) grid (5,15.000000);
\draw (0,0) -- coordinate (x axis mid) (5,0);
\draw (0,0) -- coordinate (y axis mid) (0,15.000000);
\node[h1] at (0,0) {\cross};
\draw[h1] (0,0.000000) -- (1,1.000000);
\node[h1] at(1,1.000000) {\cross};
\draw[h1] (1,1.000000) -- (2,1.000000);
\node[h1] at(2,1.000000) {\cross};
\draw[h1] (2,1.000000) -- (3,5.323262);
\node[h1] at(3,5.323262) {\cross};
\draw[h1] (3,5.323262) -- (4,6.497872);
\node[h1] at(4,6.497872) {\cross};
\draw[h1] (4,6.497872) -- (5,14.178453);
\node[h1] at(5,14.178453) {\cross};

    \end{tikzpicture}}
    \caption{Comparaison de $H_0$ et de $H_1$: test \no2}
    \label{fig:comp_2}
  \end{figure}


\section{Test \no3: Mobiles positionnés aléatoirement}
  \begin{listing}[H]
    \textfile{../tests/test_3/test_3.data}
    \caption{test\_3.data}
  \end{listing}

  \begin{figure}[H]
    \begin{center}
      \boxed{
      \begin{tikzpicture}[scale=0.5]
        \draw[grided,step=1.0,thin] (-2.000000,-8.000000) grid (14.363306,9.297633);
\draw (-2.000000,0) -- coordinate (x axis mid) (14.363306,0);
\draw (0,-8.000000) -- coordinate (y axis mid) (0,9.297633);
\foreach \x in {-2,...,14}
  \draw (\x,1pt) -- (\x,-3pt) node[anchor=north] {\x};
\foreach \y in {-8,...,9}
  \draw (1pt,\y) -- (-3pt,\y) node[anchor=east] {\y};
\node[interceptor] (I0) at (0.000000,0.000000) {\interceptor};
\node[mobile,anchor=center] (M0) at (2.000000,2.000000) {\mobile};
\node[mobile] at (M0.south east) {$M_0$};
\draw[speed] (M0.center) -- ($ (M0.center) + (-1.000000,0.000000) $);
\node[mobile,anchor=center] (M1) at (-1.000000,-1.000000) {\mobile};
\node[mobile] at (M1.south east) {$M_1$};
\draw[speed] (M1.center) -- ($ (M1.center) + (0.000000,2.000000) $);
\node[mobile,anchor=center] (M2) at (4.000000,5.000000) {\mobile};
\node[mobile] at (M2.south east) {$M_2$};
\draw[speed] (M2.center) -- ($ (M2.center) + (-1.000000,-0.300000) $);
\node[mobile,anchor=center] (M3) at (2.000000,4.000000) {\mobile};
\node[mobile] at (M3.south east) {$M_3$};
\draw[speed] (M3.center) -- ($ (M3.center) + (0.750000,-0.690000) $);
\draw[camino] (M0) -- (0.902832,2.000000);
\draw[interceptor] (0.000000,0.000000) -- (0.902832,2.000000);
\node[interceptor] at (0.902832,2.000000) {\mobile};
\node[caught] at (M0) {\mobile};
\draw[camino] (M2) -- (1.685421,4.305626);
\draw[interceptor] (0.902832,2.000000) -- (1.685421,4.305626);
\node[interceptor] at (1.685421,4.305626) {\mobile};
\node[caught] at (M2) {\mobile};
\draw[camino] (M1) -- (-1.000000,9.297633);
\draw[interceptor] (1.685421,4.305626) -- (-1.000000,9.297633);
\node[interceptor] at (-1.000000,9.297633) {\mobile};
\node[caught] at (M1) {\mobile};
\draw[camino] (M3) -- (14.363306,-7.374242);
\draw[interceptor] (-1.000000,9.297633) -- (14.363306,-7.374242);
\node[interceptor] at (14.363306,-7.374242) {\mobile};
\node[caught] at (M3) {\mobile};
\draw[interceptor] (14.363306,-7.374242) node[anchor=south east] {$t=16.484 \text{ u.t.}$};

      \end{tikzpicture}}
    \end{center}
    \caption{Heuristique $H_0$: Test \no3}
    \label{fig:H0_3}
  \end{figure}

  \begin{table}[H]
    \centering
    \begin{tabular}{|c|c|c|}
  \hline\textbf{\No mobile} & \textbf{Position interception} & \textbf{Date interception (u.t.)} \\ \hline 
  0	& $\coord{0.461}{2.000}$	 & $1.0262$ \\ \hline
  1	& $\coord{-1.000}{2.652}$	 & $1.8260$ \\ \hline
  2	& $\coord{0.959}{4.088}$	 & $3.0406$ \\ \hline
  3	& $\coord{5.797}{-0.367}$	 & $6.3288$ \\ \hline
\end{tabular}

    \caption{Heuristique $H_0$: Résultats test \no3}
    \label{tab:H0_3}
  \end{table}

  \begin{figure}[H]
    \begin{center}
      \boxed{
      \begin{tikzpicture}[scale=0.5]
        \draw[grided,step=1.0,thin] (-13.000000,-9.000000) grid (8.360891,9.000000);
\draw (-13.000000,0) -- coordinate (x axis mid) (8.360891,0);
\draw (0,-9.000000) -- coordinate (y axis mid) (0,9.000000);
\foreach \x in {-13,...,8}
  \draw (\x,1pt) -- (\x,-3pt) node[anchor=north] {\x};
\foreach \y in {-9,...,9}
  \draw (1pt,\y) -- (-3pt,\y) node[anchor=east] {\y};
\node[interceptor] (I0) at (2.000000,9.000000) {\interceptor};
\node[mobile,anchor=center] (M0) at (7.140297,8.876400) {\mobile};
\node[mobile] at (M0.south east) {$M_0$};
\node[mobile,anchor=center] (M1) at (4.767767,5.291864) {\mobile};
\node[mobile] at (M1.south east) {$M_1$};
\draw[speed] (M1.center) -- ($ (M1.center) + (-0.811979,-0.290230) $);
\node[mobile,anchor=center] (M2) at (-7.277143,-4.215198) {\mobile};
\node[mobile] at (M2.south east) {$M_2$};
\draw[speed] (M2.center) -- ($ (M2.center) + (0.458222,0.901331) $);
\node[mobile,anchor=center] (M3) at (-9.317039,-8.399123) {\mobile};
\node[mobile] at (M3.south east) {$M_3$};
\draw[speed] (M3.center) -- ($ (M3.center) + (-0.209148,0.404874) $);
\node[mobile,anchor=center] (M4) at (8.360891,5.869070) {\mobile};
\node[mobile] at (M4.south east) {$M_4$};
\draw[speed] (M4.center) -- ($ (M4.center) + (-0.505128,0.038969) $);
\node[mobile,anchor=center] (M5) at (2.371453,-8.866265) {\mobile};
\node[mobile] at (M5.south east) {$M_5$};
\draw[speed] (M5.center) -- ($ (M5.center) + (-0.225212,0.200970) $);
\draw[interceptor] (2.000000,9.000000) -- (7.140297,8.876400);
\node[interceptor] at (7.140297,8.876400) {\mobile};
\node[caught] at (M0) {\mobile};
\draw[camino] (M1) -- (-1.586399,3.020660);
\draw[interceptor] (7.140297,8.876400) -- (-1.586399,3.020660);
\node[interceptor] at (-1.586399,3.020660) {\mobile};
\node[caught] at (M1) {\mobile};
\draw[camino] (M2) -- (-3.277717,3.651746);
\draw[interceptor] (-1.586399,3.020660) -- (-3.277717,3.651746);
\node[interceptor] at (-3.277717,3.651746) {\mobile};
\node[caught] at (M2) {\mobile};
\draw[camino] (M3) -- (-12.292086,-2.639952);
\draw[interceptor] (-3.277717,3.651746) -- (-12.292086,-2.639952);
\node[interceptor] at (-12.292086,-2.639952) {\mobile};
\node[caught] at (M3) {\mobile};
\draw[camino] (M4) -- (-2.280671,6.690032);
\draw[interceptor] (-12.292086,-2.639952) -- (-2.280671,6.690032);
\node[interceptor] at (-2.280671,6.690032) {\mobile};
\node[caught] at (M4) {\mobile};
\draw[camino] (M5) -- (-3.539516,-3.591557);
\draw[interceptor] (-2.280671,6.690032) -- (-3.539516,-3.591557);
\node[interceptor] at (-3.539516,-3.591557) {\mobile};
\node[caught] at (M5) {\mobile};
\draw[interceptor] (-3.539516,-3.591557) node[anchor=south east] {$t=26.246 \text{ u.t.}$};

      \end{tikzpicture}}
    \end{center}
    \caption{Heuristique $H_1$: Test \no3}
    \label{fig:H1_3}
  \end{figure}

  \begin{table}[H]
    \centering
    \begin{tabular}{|c|c|c|}
  \hline\textbf{\No mobile} & \textbf{Position interception} & \textbf{Date interception (u.t.)} \\ \hline 
  0	& $\coord{0.000}{2.000}$	 & $1.0000$ \\ \hline
  1	& $\coord{0.000}{2.000}$	 & $1.0000$ \\ \hline
  2	& $\coord{7.323}{6.597}$	 & $5.3233$ \\ \hline
  3	& $\coord{7.249}{4.249}$	 & $6.4979$ \\ \hline
  4	& $\coord{-8.089}{5.089}$	 & $14.1785$ \\ \hline
\end{tabular}

    \caption{Heuristique $H_1$: Résultats test \no3}
    \label{tab:H1_3}
  \end{table}

  \begin{figure}[H]
    \centering
    \boxed{
    \begin{tikzpicture}[yscale=0.2]
      \draw[grided,step=1.0,thin] (0,0) grid (5,13.000000);
\draw (0,0) -- coordinate (x axis mid) (5,0);
\draw (0,0) -- coordinate (y axis mid) (0,13.000000);
\foreach \x in {0,...,5}
  \draw (\x,1pt) -- (\x,-3pt) node[anchor=north] {\x};
\node[h0] at (1,7.500000) {$H_0$};
\node[h1] at (1,5.500000) {$H_1$};
\foreach \y in {0,1,...,13}
  \draw (1pt,\y) -- (-3pt,\y) node[anchor=east] {\y};
\draw (2.500000,-2) node[anchor=north] {Nombre de mobiles interceptés};
\draw (-0.75,6.500000) node[rotate=90,anchor=south] {Temps nécessaire (u.t)};
\node[h0] at (0,0) {\cross};
\draw[h0] (0,0.000000) -- (1,1.000000);
\node[h0] at(1,1.000000) {\cross};
\draw[h0] (1,1.000000) -- (2,1.000000);
\node[h0] at(2,1.000000) {\cross};
\draw[h0] (2,1.000000) -- (3,2.769642);
\node[h0] at(3,2.769642) {\cross};
\draw[h0] (3,2.769642) -- (4,9.018394);
\node[h0] at(4,9.018394) {\cross};
\draw[h0] (4,9.018394) -- (5,12.344612);
\node[h0] at(5,12.344612) {\cross};
\draw[grided,step=1.0,thin] (0,0) grid (5,15.000000);
\draw (0,0) -- coordinate (x axis mid) (5,0);
\draw (0,0) -- coordinate (y axis mid) (0,15.000000);
\node[h1] at (0,0) {\cross};
\draw[h1] (0,0.000000) -- (1,1.000000);
\node[h1] at(1,1.000000) {\cross};
\draw[h1] (1,1.000000) -- (2,1.000000);
\node[h1] at(2,1.000000) {\cross};
\draw[h1] (2,1.000000) -- (3,5.323262);
\node[h1] at(3,5.323262) {\cross};
\draw[h1] (3,5.323262) -- (4,6.497872);
\node[h1] at(4,6.497872) {\cross};
\draw[h1] (4,6.497872) -- (5,14.178453);
\node[h1] at(5,14.178453) {\cross};

    \end{tikzpicture}}
    \caption{Comparaison de $H_0$ et de $H_1$: test \no3}
    \label{fig:comp_3}
  \end{figure}

  \section{Test \no4: Heuristique $H_1$ plus rapide}
  \begin{listing}[H]
    \textfile{../tests/test_4/test_4.data}
    \caption{test\_4.data}
  \end{listing}

  \begin{figure}[H]
    \begin{center}
      \boxed{
      \begin{tikzpicture}[scale=0.5]
        \draw[grided,step=1.0,thin] (-2.000000,-8.000000) grid (14.363306,9.297633);
\draw (-2.000000,0) -- coordinate (x axis mid) (14.363306,0);
\draw (0,-8.000000) -- coordinate (y axis mid) (0,9.297633);
\foreach \x in {-2,...,14}
  \draw (\x,1pt) -- (\x,-3pt) node[anchor=north] {\x};
\foreach \y in {-8,...,9}
  \draw (1pt,\y) -- (-3pt,\y) node[anchor=east] {\y};
\node[interceptor] (I0) at (0.000000,0.000000) {\interceptor};
\node[mobile,anchor=center] (M0) at (2.000000,2.000000) {\mobile};
\node[mobile] at (M0.south east) {$M_0$};
\draw[speed] (M0.center) -- ($ (M0.center) + (-1.000000,0.000000) $);
\node[mobile,anchor=center] (M1) at (-1.000000,-1.000000) {\mobile};
\node[mobile] at (M1.south east) {$M_1$};
\draw[speed] (M1.center) -- ($ (M1.center) + (0.000000,2.000000) $);
\node[mobile,anchor=center] (M2) at (4.000000,5.000000) {\mobile};
\node[mobile] at (M2.south east) {$M_2$};
\draw[speed] (M2.center) -- ($ (M2.center) + (-1.000000,-0.300000) $);
\node[mobile,anchor=center] (M3) at (2.000000,4.000000) {\mobile};
\node[mobile] at (M3.south east) {$M_3$};
\draw[speed] (M3.center) -- ($ (M3.center) + (0.750000,-0.690000) $);
\draw[camino] (M0) -- (0.902832,2.000000);
\draw[interceptor] (0.000000,0.000000) -- (0.902832,2.000000);
\node[interceptor] at (0.902832,2.000000) {\mobile};
\node[caught] at (M0) {\mobile};
\draw[camino] (M2) -- (1.685421,4.305626);
\draw[interceptor] (0.902832,2.000000) -- (1.685421,4.305626);
\node[interceptor] at (1.685421,4.305626) {\mobile};
\node[caught] at (M2) {\mobile};
\draw[camino] (M1) -- (-1.000000,9.297633);
\draw[interceptor] (1.685421,4.305626) -- (-1.000000,9.297633);
\node[interceptor] at (-1.000000,9.297633) {\mobile};
\node[caught] at (M1) {\mobile};
\draw[camino] (M3) -- (14.363306,-7.374242);
\draw[interceptor] (-1.000000,9.297633) -- (14.363306,-7.374242);
\node[interceptor] at (14.363306,-7.374242) {\mobile};
\node[caught] at (M3) {\mobile};
\draw[interceptor] (14.363306,-7.374242) node[anchor=south east] {$t=16.484 \text{ u.t.}$};

      \end{tikzpicture}}
    \end{center}
    \caption{Heuristique $H_0$: Test \no4}
    \label{fig:H0_4}
  \end{figure}

  \begin{table}[H]
    \centering
    \begin{tabular}{|c|c|c|}
  \hline\textbf{\No mobile} & \textbf{Position interception} & \textbf{Date interception (u.t.)} \\ \hline 
  0	& $\coord{0.461}{2.000}$	 & $1.0262$ \\ \hline
  1	& $\coord{-1.000}{2.652}$	 & $1.8260$ \\ \hline
  2	& $\coord{0.959}{4.088}$	 & $3.0406$ \\ \hline
  3	& $\coord{5.797}{-0.367}$	 & $6.3288$ \\ \hline
\end{tabular}

    \caption{Heuristique $H_0$: Résultats test \no4}
    \label{tab:H0_4}
  \end{table}

  \begin{figure}[H]
    \begin{center}
      \boxed{
      \begin{tikzpicture}[scale=1]
        \draw[grided,step=1.0,thin] (-13.000000,-9.000000) grid (8.360891,9.000000);
\draw (-13.000000,0) -- coordinate (x axis mid) (8.360891,0);
\draw (0,-9.000000) -- coordinate (y axis mid) (0,9.000000);
\foreach \x in {-13,...,8}
  \draw (\x,1pt) -- (\x,-3pt) node[anchor=north] {\x};
\foreach \y in {-9,...,9}
  \draw (1pt,\y) -- (-3pt,\y) node[anchor=east] {\y};
\node[interceptor] (I0) at (2.000000,9.000000) {\interceptor};
\node[mobile,anchor=center] (M0) at (7.140297,8.876400) {\mobile};
\node[mobile] at (M0.south east) {$M_0$};
\node[mobile,anchor=center] (M1) at (4.767767,5.291864) {\mobile};
\node[mobile] at (M1.south east) {$M_1$};
\draw[speed] (M1.center) -- ($ (M1.center) + (-0.811979,-0.290230) $);
\node[mobile,anchor=center] (M2) at (-7.277143,-4.215198) {\mobile};
\node[mobile] at (M2.south east) {$M_2$};
\draw[speed] (M2.center) -- ($ (M2.center) + (0.458222,0.901331) $);
\node[mobile,anchor=center] (M3) at (-9.317039,-8.399123) {\mobile};
\node[mobile] at (M3.south east) {$M_3$};
\draw[speed] (M3.center) -- ($ (M3.center) + (-0.209148,0.404874) $);
\node[mobile,anchor=center] (M4) at (8.360891,5.869070) {\mobile};
\node[mobile] at (M4.south east) {$M_4$};
\draw[speed] (M4.center) -- ($ (M4.center) + (-0.505128,0.038969) $);
\node[mobile,anchor=center] (M5) at (2.371453,-8.866265) {\mobile};
\node[mobile] at (M5.south east) {$M_5$};
\draw[speed] (M5.center) -- ($ (M5.center) + (-0.225212,0.200970) $);
\draw[interceptor] (2.000000,9.000000) -- (7.140297,8.876400);
\node[interceptor] at (7.140297,8.876400) {\mobile};
\node[caught] at (M0) {\mobile};
\draw[camino] (M1) -- (-1.586399,3.020660);
\draw[interceptor] (7.140297,8.876400) -- (-1.586399,3.020660);
\node[interceptor] at (-1.586399,3.020660) {\mobile};
\node[caught] at (M1) {\mobile};
\draw[camino] (M2) -- (-3.277717,3.651746);
\draw[interceptor] (-1.586399,3.020660) -- (-3.277717,3.651746);
\node[interceptor] at (-3.277717,3.651746) {\mobile};
\node[caught] at (M2) {\mobile};
\draw[camino] (M3) -- (-12.292086,-2.639952);
\draw[interceptor] (-3.277717,3.651746) -- (-12.292086,-2.639952);
\node[interceptor] at (-12.292086,-2.639952) {\mobile};
\node[caught] at (M3) {\mobile};
\draw[camino] (M4) -- (-2.280671,6.690032);
\draw[interceptor] (-12.292086,-2.639952) -- (-2.280671,6.690032);
\node[interceptor] at (-2.280671,6.690032) {\mobile};
\node[caught] at (M4) {\mobile};
\draw[camino] (M5) -- (-3.539516,-3.591557);
\draw[interceptor] (-2.280671,6.690032) -- (-3.539516,-3.591557);
\node[interceptor] at (-3.539516,-3.591557) {\mobile};
\node[caught] at (M5) {\mobile};
\draw[interceptor] (-3.539516,-3.591557) node[anchor=south east] {$t=26.246 \text{ u.t.}$};

      \end{tikzpicture}}
    \end{center}
    \caption{Heuristique $H_1$: Test \no4}
    \label{fig:H1_4}
  \end{figure}

  \begin{table}[H]
    \centering
    \begin{tabular}{|c|c|c|}
  \hline\textbf{\No mobile} & \textbf{Position interception} & \textbf{Date interception (u.t.)} \\ \hline 
  0	& $\coord{0.000}{2.000}$	 & $1.0000$ \\ \hline
  1	& $\coord{0.000}{2.000}$	 & $1.0000$ \\ \hline
  2	& $\coord{7.323}{6.597}$	 & $5.3233$ \\ \hline
  3	& $\coord{7.249}{4.249}$	 & $6.4979$ \\ \hline
  4	& $\coord{-8.089}{5.089}$	 & $14.1785$ \\ \hline
\end{tabular}

    \caption{Heuristique $H_1$: Résultats test \no4}
    \label{tab:H1_4}
  \end{table}

  \begin{figure}[H]
    \centering
    \boxed{
    \begin{tikzpicture}[yscale=0.35]
      \draw[grided,step=1.0,thin] (0,0) grid (5,13.000000);
\draw (0,0) -- coordinate (x axis mid) (5,0);
\draw (0,0) -- coordinate (y axis mid) (0,13.000000);
\foreach \x in {0,...,5}
  \draw (\x,1pt) -- (\x,-3pt) node[anchor=north] {\x};
\node[h0] at (1,7.500000) {$H_0$};
\node[h1] at (1,5.500000) {$H_1$};
\foreach \y in {0,1,...,13}
  \draw (1pt,\y) -- (-3pt,\y) node[anchor=east] {\y};
\draw (2.500000,-2) node[anchor=north] {Nombre de mobiles interceptés};
\draw (-0.75,6.500000) node[rotate=90,anchor=south] {Temps nécessaire (u.t)};
\node[h0] at (0,0) {\cross};
\draw[h0] (0,0.000000) -- (1,1.000000);
\node[h0] at(1,1.000000) {\cross};
\draw[h0] (1,1.000000) -- (2,1.000000);
\node[h0] at(2,1.000000) {\cross};
\draw[h0] (2,1.000000) -- (3,2.769642);
\node[h0] at(3,2.769642) {\cross};
\draw[h0] (3,2.769642) -- (4,9.018394);
\node[h0] at(4,9.018394) {\cross};
\draw[h0] (4,9.018394) -- (5,12.344612);
\node[h0] at(5,12.344612) {\cross};
\draw[grided,step=1.0,thin] (0,0) grid (5,15.000000);
\draw (0,0) -- coordinate (x axis mid) (5,0);
\draw (0,0) -- coordinate (y axis mid) (0,15.000000);
\node[h1] at (0,0) {\cross};
\draw[h1] (0,0.000000) -- (1,1.000000);
\node[h1] at(1,1.000000) {\cross};
\draw[h1] (1,1.000000) -- (2,1.000000);
\node[h1] at(2,1.000000) {\cross};
\draw[h1] (2,1.000000) -- (3,5.323262);
\node[h1] at(3,5.323262) {\cross};
\draw[h1] (3,5.323262) -- (4,6.497872);
\node[h1] at(4,6.497872) {\cross};
\draw[h1] (4,6.497872) -- (5,14.178453);
\node[h1] at(5,14.178453) {\cross};

    \end{tikzpicture}}
    \caption{Comparaison de $H_0$ et de $H_1$: test \no4}
    \label{fig:comp_4}
  \end{figure}
