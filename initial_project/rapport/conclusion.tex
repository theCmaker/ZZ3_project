Le but de ce projet était de maximiser le nombre de mobiles interceptés en minimisant le temps d'interception.

Nous avons donc construit deux heuristiques capables de fournir une solution stable et d'interpréter les résultats.

On remarque que les heuristiques tendent vers des temps d'interceptions similaires lorsque l'on essaye d'intercepter tous les mobiles mais que pour un nombre faible de mobiles, l'heuristique $H_0$ semble être plus efficace.

Cependant, elle est plus coûteuse en temps CPU que l'heuristique $H_1$ puisqu'elle demande de recalculer le temps d'interception minimum à chaque itération.

Enfin ces deux solutions ne fournissent pas la solution optimale et donc, il est possible de définir des variantes qui pourraient obtenir de meilleurs résultats.

\section*{Proposition d'améliorations}

On pourrait considérer que l'intercepteur se déplace à une vitesse variable (majorée par une vitesse max $v_1$) et chercher à trouver la vitesse $v$ qui nous permettrait d'intercepter un mobile plus rapidement.

Afin de trouver la séquence idéale, minimisant le temps de parcours, on peut partir d'une séquence initiale, lui appliquer des opérations particulières (intervertion de mobiles/ajout/suppression), et comparer les différents parcours possibles pour trouver le meilleur compromis.

On pourrait également concevoir une gestion de priorité au niveau des mobiles (par exemple pour inspecter des zones en priorité avec des drônes).

Il serait ainsi possible de proposer des heuristiques afin d'optimiser un rapport mobiles interceptés / temps de parcours ou bien mobiles / distance parcourue.
