\documentclass{report}
\usepackage[utf8]{inputenc} %encodage entrée
\usepackage[T1]{fontenc}
\usepackage{graphicx} %images
\usepackage[usenames,dvipsnames]{xcolor} %couleurs
\usepackage{tikz} %schémas
\usepackage{algo} %mise en forme d'algos
\usepackage{minted} %mise en forme de code source
\usepackage{framed} %cadres et bordures
\usepackage[frenchb]{babel} %langue
\usepackage{amsmath} %symboles maths
\usepackage{caption} %legendes
\usepackage{subcaption} %légendes et sous-figures
\usepackage{enumitem} %formatage des listes à puces
\usepackage[nobottomtitles]{titlesec} %formatage des chapitres
\usepackage{blindtext}
\usepackage[a4paper]{geometry} %mise en page
\usepackage[hidelinks]{hyperref} %liens
\usepackage{lastpage} %pagination 1/n
\usepackage{fancyhdr} %headers&footers

\input{settings/colors}
\input{settings/algo}
\input{settings/mintedC}
\input{settings/mintedMakefile}
\input{settings/mintedBash}
\input{settings/hyperrefSettings}

\hypersetup{
  pdftitle={Projet ZZ1 - Interception de mobiles}
}

%styles et formatage
\geometry{scale=0.8,centering}
\frenchbsetup{StandardLists=true}
\newcommand{\hsp}{\hspace{20pt}}
\newcommand{\blankpage}{\newpage \thispagestyle{empty} \addtocounter{page}{-1} \null \newpage}
\titleformat{\chapter}[hang]{\LARGE\bfseries}{\thechapter\hsp\textcolor{lightgray}{|}\hsp}{0pt}{\LARGE\bfseries}

% Pages de contenu
\fancypagestyle{IHA-fancy-style}{%
  \fancyhf{}%
  \fancyhead[R]{\leftmark}
  \fancyfoot[R]{\thepage/\pageref{LastPage}}
  \fancyfoot[L]{Interception de mobiles}%
  \renewcommand{\headrulewidth}{0.4pt}% Ligne de header
  \renewcommand{\footrulewidth}{0.4pt}% Ligne de footer
}
% Style de base: sert pour les nouveaux chapitres
\fancypagestyle{plain}{%
  \fancyhf{}%
  \fancyfoot[R]{\thepage/\pageref{LastPage}}%
  \fancyfoot[L]{Interception de mobiles}%
  \renewcommand{\headrulewidth}{0pt}% pas de ligne de Header
  \renewcommand{\footrulewidth}{0.4pt}% ligne de footer
}
\pagestyle{IHA-fancy-style}


\begin{document}
  \begin{titlepage}
  \newcommand{\HRule}{\rule{\linewidth}{0.5mm}}
  \center
  \null{}
  \vspace{1cm}

  \textsc{\LARGE ISIMA Première Année}\\[1.5cm]
  \textsc{\Large Projet}\\[1.5cm]
  \HRule \\[0.4cm]
  { \huge \bfseries Interception de mobiles}\\
  \HRule \\[1.5cm]

  \begin{center}
    \boxed{
      \begin{tikzpicture}[scale=1]
        \draw[grided,step=1.0,thin] (-1.000000,-2.000000) grid (7.784527,5.000000);
\draw (-1.000000,0) -- coordinate (x axis mid) (7.784527,0);
\draw (0,-2.000000) -- coordinate (y axis mid) (0,5.000000);
\foreach \x in {-1,...,7}
  \draw (\x,1pt) -- (\x,-3pt) node[anchor=north] {\x};
\foreach \y in {-2,...,5}
  \draw (1pt,\y) -- (-3pt,\y) node[anchor=east] {\y};
\node[interceptor] (I0) at (0.000000,0.000000) {\interceptor};
\node[mobile,anchor=center] (M0) at (2.000000,2.000000) {\mobile};
\node[mobile] at (M0.south east) {$M_0$};
\draw[speed] (M0.center) -- ($ (M0.center) + (-1.000000,0.000000) $);
\node[mobile,anchor=center] (M1) at (-1.000000,-1.000000) {\mobile};
\node[mobile] at (M1.south east) {$M_1$};
\draw[speed] (M1.center) -- ($ (M1.center) + (0.000000,2.000000) $);
\node[mobile,anchor=center] (M2) at (4.000000,5.000000) {\mobile};
\node[mobile] at (M2.south east) {$M_2$};
\draw[speed] (M2.center) -- ($ (M2.center) + (-1.000000,-0.300000) $);
\node[mobile,anchor=center] (M3) at (2.000000,4.000000) {\mobile};
\node[mobile] at (M3.south east) {$M_3$};
\draw[speed] (M3.center) -- ($ (M3.center) + (0.750000,-0.690000) $);
\draw[camino] (M0) -- (0.902832,2.000000);
\draw[interceptor] (0.000000,0.000000) -- (0.902832,2.000000);
\node[interceptor] at (0.902832,2.000000) {\mobile};
\node[caught] at (M0) {\mobile};
\draw[camino] (M1) -- (-1.000000,3.844238);
\draw[interceptor] (0.902832,2.000000) -- (-1.000000,3.844238);
\node[interceptor] at (-1.000000,3.844238) {\mobile};
\node[caught] at (M1) {\mobile};
\draw[camino] (M2) -- (0.715771,4.014731);
\draw[interceptor] (-1.000000,3.844238) -- (0.715771,4.014731);
\node[interceptor] at (0.715771,4.014731) {\mobile};
\node[caught] at (M2) {\mobile};
\draw[camino] (M3) -- (7.784527,-1.321765);
\draw[interceptor] (0.715771,4.014731) -- (7.784527,-1.321765);
\node[interceptor] at (7.784527,-1.321765) {\mobile};
\node[caught] at (M3) {\mobile};
\draw[interceptor] (7.784527,-1.321765) node[anchor=south east] {$t=7.713 \text{ u.t.}$};

      \end{tikzpicture}
    }
  \end{center}
  \vfill

  \begin{minipage}{0.4\textwidth}
    \begin{flushleft} \large
      Axel DELSOL\\
      Pierre-Loup PISSAVY\\
    \end{flushleft}
  \end{minipage}
  ~
  \begin{minipage}{0.4\textwidth}
    \begin{flushright} \large
      \emph{Tuteur de projet :} \\
      Christophe DUHAMEL
    \end{flushright}
  \end{minipage}\\[1cm]

  {\large mars -- juin 2015}\\[1cm]

  \vfill

  \includegraphics[width=6cm]{settings/ISIMA_logo.pdf}\\[1cm]
\end{titlepage}

  \blankpage
  \tableofcontents
  \blankpage
  %redefinition des ecarts apres table des matieres
  \setlength{\parskip}{10pt}
  \setlength{\parindent}{0pt}
  \chapter{Présentation}

  \chapter{Travail effectué}

  \chapter{Conclusion}

\end{document}
