\documentclass{report}
\usepackage[utf8]{inputenc} %encodage entrée
\usepackage[T1]{fontenc}
\usepackage{graphicx} %images
\usepackage[usenames,dvipsnames]{xcolor} %couleurs
\usepackage{tikz} %schémas
\usetikzlibrary{calc}
\usepackage{algo} %mise en forme d'algos
\usepackage[chapter]{minted} %mise en forme de code source
\usepackage{framed} %cadres et bordures
\usepackage[frenchb]{babel} %langue
\usepackage{amsmath} %symboles maths
\usepackage{caption} %legendes
\usepackage{subcaption} %légendes et sous-figures
\usepackage{enumitem} %formatage des listes à puces
\usepackage[nobottomtitles]{titlesec} %formatage des chapitres
\usepackage{blindtext}
\usepackage[a4paper]{geometry} %mise en page
\usepackage[hidelinks]{hyperref} %liens
\usepackage{lastpage} %pagination 1/n
\usepackage{fancyhdr} %headers&footers

%intervals
\newcommand{\interval}[4]{\mathopen{#1}#2\mathclose{}\mathpunct{};#3\mathclose{#4}}
\newcommand{\intervalcc}[2]{\interval{[}{#1}{#2}{]}}
\newcommand{\intervaloc}[2]{\interval{]}{#1}{#2}{]}}
\newcommand{\intervalco}[2]{\interval{[}{#1}{#2}{[}}
\newcommand{\intervaloo}[2]{\interval{]}{#1}{#2}{[}}
\newcommand{\coord}[2]{\interval{(}{#1}{#2}{)}}
%sets
\newcommand{\mathset}[1]{\mathbb{#1}}
\newcommand{\N}{\mathset{N}} %integers
\newcommand{\Z}{\mathset{Z}} %relative integers
\newcommand{\Q}{\mathset{Q}} %rationnals
\newcommand{\R}{\mathset{R}} %reals
\newcommand{\C}{\mathset{C}} %complex
%setas
\newcommand{\setas}[2]{\left\{#1\mathrel{}\middle|\mathrel{}#2\right\}}
%diff
\newcommand{\diff}{\mathop{}\mathopen{}\mathrm{d}}
%abs and norm
\newcommand{\abs}[1]{\left\lvert#1\right\rvert}
\newcommand{\norm}[1]{\left\lVert#1\right\rVert}
%dot product
\newcommand{\dotprod}[2]{#1\cdot#2}
\newcommand{\dotproda}[2]{\left\langle#1,#2\right\rangle}
%geometry
\newcommand{\curve}{\mathscr{C}}
%text
\newcommand{\qtext}[1]{\quad\text{#1}\quad}
\newcommand{\qqtext}[1]{\qquad\text{#1}\qquad}
\input{settings/mintedSettings}
\usepackage{MnSymbol}
\newminted{text}{
  linenos                = false,
  breaklines             = true,
  breakautoindent        = true,
  breaksymbolleft        = $\lhookrightarrow$,
  breaksymbolindentleft  = 10pt,
  breaksymbolsepleft     = 2pt,
  breaksymbolright       = $\rhookleftarrow$,
  breaksymbolindentright = 10pt,
  breaksymbolsepright    = 2pt,
  texcomments            = true
}

\newmintedfile{text}{
  linenos                = false,
  breaklines             = true,
  breakautoindent        = true,
  breaksymbolleft        = $\lhookrightarrow$,
  breaksymbolindentleft  = 10pt,
  breaksymbolsepleft     = 2pt,
  breaksymbolright       = $\rhookleftarrow$,
  breaksymbolindentright = 10pt,
  breaksymbolsepright    = 2pt,
  texcomments            = false
}

\newmintinline{text}{
}

\input{settings/hyperrefSettings}

\hypersetup{
  pdftitle={Projet ZZ1 - Interception de mobiles}
}

%styles et formatage
\geometry{scale=0.8,centering}
\frenchbsetup{StandardLists=true}
\newcommand{\hsp}{\hspace{20pt}}
\newcommand{\TikZ}{Ti{\color{orange}\textit{k}}Z}
\newcommand{\blankpage}{\newpage \thispagestyle{empty} \addtocounter{page}{-1} \null \newpage}
\titleformat{\chapter}[hang]{\LARGE\bfseries}{\thechapter\hsp\textcolor{lightgray}{|}\hsp}{0pt}{\LARGE\bfseries}

\captionsetup[table]{name=\textsc{Tableau}}
\renewcommand{\listingscaption}{Fichier}
\renewcommand{\listoflistingscaption}{Liste des fichiers}

\newcommand{\mobile}{\Large{$+$}}
\newcommand{\cross}{\Large{$\times$}}
\newcommand{\interceptor}{\Large{$\square$}}
\tikzset{%styles schémas
    interceptor/.style={thick,color=Blue},  %intercepteur
    mobile/.style={color=LimeGreen},  %mobile
    camino/.style={mobile,dashed}, %chemin mobile
    speed/.style={mobile, thick, >=latex,->}, %vitesse mobile
    caught/.style={color=OrangeRed},  %mobile attrapé
    grided/.style={dotted,color=Black!20},    %grille
    h0/.style={color=Purple},
    h1/.style={color=RoyalBlue}
}

% Pages de contenu
\fancypagestyle{IHA-fancy-style}{%
  \fancyhf{}%
  \fancyhead[R]{\leftmark}
  \fancyfoot[R]{\thepage/\pageref{LastPage}}
  \fancyfoot[L]{Projet ZZ1 -- Interception de mobiles}%
  \renewcommand{\headrulewidth}{0.4pt}% Ligne de header
  \renewcommand{\footrulewidth}{0.4pt}% Ligne de footer
}
% Style de base: sert pour les nouveaux chapitres
\fancypagestyle{plain}{%
  \fancyhf{}%
  \fancyfoot[R]{\thepage/\pageref{LastPage}}%
  \fancyfoot[L]{Projet ZZ1 -- Interception de mobiles}%
  \renewcommand{\headrulewidth}{0pt}% pas de ligne de Header
  \renewcommand{\footrulewidth}{0.4pt}% ligne de footer
}
\pagestyle{IHA-fancy-style}


\begin{document}
  \begin{titlepage}
  \newcommand{\HRule}{\rule{\linewidth}{0.5mm}}
  \center
  \null{}
  \vspace{1cm}

  \textsc{\LARGE ISIMA Première Année}\\[1.5cm]
  \textsc{\Large Projet}\\[1.5cm]
  \HRule \\[0.4cm]
  { \huge \bfseries Interception de mobiles}\\
  \HRule \\[1.5cm]

  \begin{center}
    \boxed{
      \begin{tikzpicture}[scale=1]
        \draw[grided,step=1.0,thin] (-1.000000,-2.000000) grid (7.784527,5.000000);
\draw (-1.000000,0) -- coordinate (x axis mid) (7.784527,0);
\draw (0,-2.000000) -- coordinate (y axis mid) (0,5.000000);
\foreach \x in {-1,...,7}
  \draw (\x,1pt) -- (\x,-3pt) node[anchor=north] {\x};
\foreach \y in {-2,...,5}
  \draw (1pt,\y) -- (-3pt,\y) node[anchor=east] {\y};
\node[interceptor] (I0) at (0.000000,0.000000) {\interceptor};
\node[mobile,anchor=center] (M0) at (2.000000,2.000000) {\mobile};
\node[mobile] at (M0.south east) {$M_0$};
\draw[speed] (M0.center) -- ($ (M0.center) + (-1.000000,0.000000) $);
\node[mobile,anchor=center] (M1) at (-1.000000,-1.000000) {\mobile};
\node[mobile] at (M1.south east) {$M_1$};
\draw[speed] (M1.center) -- ($ (M1.center) + (0.000000,2.000000) $);
\node[mobile,anchor=center] (M2) at (4.000000,5.000000) {\mobile};
\node[mobile] at (M2.south east) {$M_2$};
\draw[speed] (M2.center) -- ($ (M2.center) + (-1.000000,-0.300000) $);
\node[mobile,anchor=center] (M3) at (2.000000,4.000000) {\mobile};
\node[mobile] at (M3.south east) {$M_3$};
\draw[speed] (M3.center) -- ($ (M3.center) + (0.750000,-0.690000) $);
\draw[camino] (M0) -- (0.902832,2.000000);
\draw[interceptor] (0.000000,0.000000) -- (0.902832,2.000000);
\node[interceptor] at (0.902832,2.000000) {\mobile};
\node[caught] at (M0) {\mobile};
\draw[camino] (M1) -- (-1.000000,3.844238);
\draw[interceptor] (0.902832,2.000000) -- (-1.000000,3.844238);
\node[interceptor] at (-1.000000,3.844238) {\mobile};
\node[caught] at (M1) {\mobile};
\draw[camino] (M2) -- (0.715771,4.014731);
\draw[interceptor] (-1.000000,3.844238) -- (0.715771,4.014731);
\node[interceptor] at (0.715771,4.014731) {\mobile};
\node[caught] at (M2) {\mobile};
\draw[camino] (M3) -- (7.784527,-1.321765);
\draw[interceptor] (0.715771,4.014731) -- (7.784527,-1.321765);
\node[interceptor] at (7.784527,-1.321765) {\mobile};
\node[caught] at (M3) {\mobile};
\draw[interceptor] (7.784527,-1.321765) node[anchor=south east] {$t=7.713 \text{ u.t.}$};

      \end{tikzpicture}
    }
  \end{center}
  \vfill

  \begin{minipage}{0.4\textwidth}
    \begin{flushleft} \large
      Axel DELSOL\\
      Pierre-Loup PISSAVY\\
    \end{flushleft}
  \end{minipage}
  ~
  \begin{minipage}{0.4\textwidth}
    \begin{flushright} \large
      \emph{Tuteur de projet :} \\
      Christophe DUHAMEL
    \end{flushright}
  \end{minipage}\\[1cm]

  {\large mars -- juin 2015}\\[1cm]

  \vfill

  \includegraphics[width=6cm]{settings/ISIMA_logo.pdf}\\[1cm]
\end{titlepage}

  %\blankpage
  \setlength{\parskip}{10pt}
  \setlength{\parindent}{0pt}
  \tableofcontents
  %\blankpage

  \chapter{Introduction}
    \section{Problème à modéliser}
Nous disposons d'un intercepteur dont la vitesse $v_1$ est constante et de $n$ mobiles.

Il nous faut intercepter autant de mobiles que possible en un temps minimal.

\section{Méthodes utilisées}

	\subsection{Notations}
		Chaque mobile se déplace à vitesse constante $\norm{\vect{v_0}}$ selon une direction déterminée par son vecteur vitesse $\vect{v_0}$.

		On peut décomposer cette vitesse selon les axes du plan:
		\[ \vect{v_0} = \vPLANco{v_0^x}{v_0^y} \]

		On note la position initiale de l'intercepteur:
		\[
		\vect{i}(t=0) = 
		\left(
		\begin{array}{c}
		 x_1 \\
		 y_1
		\end{array}
		\right)
		\]

	\subsection{Idées}
	  Une première approche nous mène à supposer qu'il peut être intéressant d'intercepter le mobile que l'on peut atteindre le plus rapidement. 

	  Nous pouvons également tenter d'intercepter les mobiles un par un dans un ordre aléatoire. Cela revient à définir une séquence (aléatoire ou non) et à calculer le temps qu'il sera nécessaire pour les intercepter dans cet ordre (si c'est possible).

	  Dans tous les cas imaginables, il sera nécessaire de déterminer les positions successives de l'intercepteur au cours du temps ainsi que les différentes directions dans lesquelles il devra se déplacer.

\section{Outils proposés}
	Nous avons construit un ensemble de fonctions:
	\begin{itemize}
		\item Calcul de la position d'un mobile à un un instant $t$,
		\item Calcul de l'angle que doit prendre l'intercepteur pour intercepter un mobile à partir de sa position courante,
		\item Calcul de la durée nécessaire pour intercepter un mobile à partir de la position courante de l'intercepteur,
		\item Calcul de la position future de l'intercepteur à partir de sa position courante, de l'angle, et du temps nécessaire.
	\end{itemize}

	Egalement, nous avons conçu une structure de fichier permettant de fournir au programme de calcul toutes les données relatives aux mobiles (position initiale, vitesse, direction) et à l'intercepteur (position initiale, vitesse). Le format de ce fichier lui permet une évolutivité: nous avons laissé la possibilité de définir plusieurs intercepteurs pour des modifications ultérieures du calcul de parcours en impliquant plusieurs.

	Un exemple de fichier est présenté en fichier \ref{lst:exemple_graph}. Chaque ligne commençant par un croisillon (\texttt{\#}) est un commentaire qui ne sera pas interprété par le programme de calcul.

	\begin{listing}[H]
		\textfile{exemple_graph.data}
      	\caption{exemple\_graph.data}
      	\label{lst:exemple_graph}
  	\end{listing}

  	Nous avons organisé notre code de la manière suivante:
  	\begin{itemize}
  		\item Un dossier \texttt{tests} contenant plusieurs sous-dossiers, chacun correspondant à l'un des 4 tests proposés ici,
  		\item Un dossier \texttt{rapport} contenant toutes les sources du présent rapport,
  		\item Un dossier \texttt{src} contenant tous les fichiers sources,
        \item Un dossier \texttt{include} contenant toutes les déclarations de fonctions et de structures utilisées,
        \item Un dossier \texttt{bin} dans lequel sera rangé l'exécutable final.
  	\end{itemize}

  
  \chapter{Problèmes étudiés}
    \section{Calcul d'interception}

On modélise le déplacement de l'intercepteur par la fonction suivante : 
\[
\vect{i}(t,\alpha) = 
\left(
\begin{array}{c}
 x_1 + t \cdot v_1 \cdot \cos(\alpha)\\
 y_1 + t \cdot v_1 \cdot \sin(\alpha)
\end{array}
\right)
\]
avec $t \in \R^+$ et $\alpha \in \icc{-\pi}{\pi}$.

On modélise de même la même manière le déplacement du mobile :
\[
\vect{m}(t) = 
\left(
\begin{array}{c}
 x_0 + t \cdot v^{x}_0\\
 y_0 + t \cdot v^{y}_0
\end{array}
\right)
\]
avec $t \in \R^+$.

On doit donc résoudre le système d'équations suivant afin de calculer le temps d'interception d'un mobile:

\[
\left\{
\begin{array}{r c l}
x_1 + t \cdot v_1 \cdot \cos(\alpha) &=& x_0 + t \cdot v^{x}_0\\
y_1 + t \cdot v_1 \cdot \sin(\alpha) &=& y_0 + t \cdot v^{y}_0
\end{array}
\right.
\]

La valeur est donnée par la résolution de l'équation $a \cdot \cos(\alpha)+b \cdot \sin(\alpha) = c$ avec:
\[
\left\{
\begin{array}{r c l}
a &=& y_0 - y_1\\
b &=& x_1 - x_0\\
c &=& \displaystyle \frac{a \cdot v^{x}_0 +b \cdot v^{y}_0}{v_1}
\end{array}
\right.
\]

On obtient alors 2 possibilités pour la date t : 
\[ \frac{-b}{-v^{x}_0 + v_1 \cdot \cos(\alpha)}  \qqtext{et} \frac{a}{-v^{y}_0 + v_1 \cdot \sin(\alpha)} \]

La fonction d'interception teste alors les positions obtenues avec les deux dates et retient celle qui fonctionne et qui est minimale.

\section{Heuristique $H_0$}
	Nous proposons une première méthode heuristique $H_0$ qui intercepte successivement les mobiles qui seront interceptés le plus rapidement. 
	A chaque nouvelle interception, les temps d'interception pour atteindre les mobiles restants sont recalculés, et l'on conserve le plus faible d'entre-eux.

	Ainsi le premier mobile intercepté n'est pas nécessairement le plus proche. En effet, il suffit qu'il s'éloigne de la position initiale de l'intercepteur tandis qu'un autre, plus éloigné au départ s'en rapproche suffisamment vite pour que l'intercepteur commence par intercepter ce dernier.

	A chaque étape de recherche du temps d'interception minimal, nous conservons les paramètres d'orientation de l'intercepteur et le temps calculé, cela permet ainsi de limiter le nombre de calculs.

\section{Heuristique $H_1$}
	La seconde méthode heuristique que nous proposons permet, à partir d'une séquence donnée définissant l'ordre dans lequel les mobiles doivent être interceptés, de calculer le temps qui sera nécessaire pour intercepter tous les mobiles qui sont accessibles tout en respectant l'ordre demandé.

	Cette méthode demande moins de ressources de calcul que la précédente dans la mesure où l'ordre est défini par avance: il n'y a donc pas à déterminer de temps minimal.


  
  \chapter{Tests réalisés}
    Nous présentons dans cette partie un ensemble de tests que nous avons effectués pour contrôler le bon fonctionnement de nos algorithmes. 

Nous avons réalisé des graphiques afin d'avoir une meilleure interprétation des résultats obtenus.

Sur ces graphiques, nous avons choisi de représenter les mobiles $M_i$ non-interceptés par des croix vertes (\tikz[baseline=-0.5ex]{\node[mobile,inner sep=0,outer sep=0]{\mobile};}), et les mobiles $M_i$ interceptés par des croix rouges (\tikz[baseline=-0.5ex]{\node[caught,inner sep=0,outer sep=0]{\mobile};}).

Les vecteurs vitesse et la trajectoire empruntée par les mobiles sont indiqués en vert par des vecteurs (\tikz[baseline=-0.5ex]{\draw[speed] (0,0) -- (1,0);}) et des lignes pointillées (\tikz[baseline=-0.5ex]{\draw[camino] (0,0) -- (1,0);}).

La position initiale de l'intercepteur est repérée par un carré bleu (\tikz[baseline=-0.5ex]{\node[interceptor,inner sep=0,outer sep=0]{\interceptor};}) et ses positions successives par des croix bleues (\tikz[baseline=-0.5ex]{\node[interceptor,inner sep=0,outer sep=0]{\mobile};}). La date de la dernière interception est indiquée au-dessus de la position où elle a lieu.

\section{Test \no1: Tous les mobiles interceptés, séquences différentes}
  \textfile{../tests/test_1/test_1.data}

  \begin{figure}[H]
    \begin{center}
      \boxed{
      \begin{tikzpicture}[scale=0.7]
        \draw[grided,step=1.0,thin] (-3.000000,-2.000000) grid (14.344612,8.703384);
\draw (-3.000000,0) -- coordinate (x axis mid) (14.344612,0);
\draw (0,-2.000000) -- coordinate (y axis mid) (0,8.703384);
\foreach \x in {-3,...,14}
  \draw (\x,1pt) -- (\x,-3pt) node[anchor=north] {\x};
\foreach \y in {-2,...,8}
  \draw (1pt,\y) -- (-3pt,\y) node[anchor=east] {\y};
\node[interceptor] (I0) at (0.000000,0.000000) {\interceptor};
\node[mobile,anchor=center] (M0) at (1.000000,2.000000) {\mobile};
\node[mobile] at (M0.south east) {$M_0$};
\draw[speed] (M0.center) -- ($ (M0.center) + (-1.000000,0.000000) $);
\node[mobile,anchor=center] (M1) at (0.000000,3.000000) {\mobile};
\node[mobile] at (M1.south east) {$M_1$};
\draw[speed] (M1.center) -- ($ (M1.center) + (0.000000,-1.000000) $);
\node[mobile,anchor=center] (M2) at (2.000000,5.000000) {\mobile};
\node[mobile] at (M2.south east) {$M_2$};
\draw[speed] (M2.center) -- ($ (M2.center) + (1.000000,0.300000) $);
\node[mobile,anchor=center] (M3) at (4.000000,1.000000) {\mobile};
\node[mobile] at (M3.south east) {$M_3$};
\draw[speed] (M3.center) -- ($ (M3.center) + (0.500000,0.500000) $);
\node[mobile,anchor=center] (M4) at (-1.000000,-2.000000) {\mobile};
\node[mobile] at (M4.south east) {$M_4$};
\draw[speed] (M4.center) -- ($ (M4.center) + (-0.500000,0.500000) $);
\draw[camino] (M0) -- (0.000000,2.000000);
\draw[interceptor] (0.000000,0.000000) -- (0.000000,2.000000);
\node[interceptor] at (0.000000,2.000000) {\mobile};
\node[caught] at (M0) {\mobile};
\draw[camino] (M1) -- (0.000000,2.000000);
\draw[interceptor] (0.000000,2.000000) -- (0.000000,2.000000);
\node[interceptor] at (0.000000,2.000000) {\mobile};
\node[caught] at (M1) {\mobile};
\draw[camino] (M4) -- (-2.384821,-0.615179);
\draw[interceptor] (0.000000,2.000000) -- (-2.384821,-0.615179);
\node[interceptor] at (-2.384821,-0.615179) {\mobile};
\node[caught] at (M4) {\mobile};
\draw[camino] (M3) -- (8.509197,5.509197);
\draw[interceptor] (-2.384821,-0.615179) -- (8.509197,5.509197);
\node[interceptor] at (8.509197,5.509197) {\mobile};
\node[caught] at (M3) {\mobile};
\draw[camino] (M2) -- (14.344612,8.703384);
\draw[interceptor] (8.509197,5.509197) -- (14.344612,8.703384);
\node[interceptor] at (14.344612,8.703384) {\mobile};
\node[caught] at (M2) {\mobile};
\draw[interceptor] (14.344612,8.703384) node[anchor=south east] {$t=12.345 \text{ u.t.}$};

      \end{tikzpicture}}
    \end{center}
    \caption{Heuristique $H_0$: Test \no1}
    \label{fig:H0_1}
  \end{figure}

  \begin{table}[H]
    \centering
    \begin{tabular}{|c|c|c|}
  \hline\textbf{\No mobile} & \textbf{Position interception} & \textbf{Date interception (u.t.)} \\ \hline 
  0	& $\coord{0.000}{2.000}$	 & $1.0000$ \\ \hline
  1	& $\coord{0.000}{2.000}$	 & $1.0000$ \\ \hline
  4	& $\coord{-2.385}{-0.615}$	 & $2.7696$ \\ \hline
  3	& $\coord{8.509}{5.509}$	 & $9.0184$ \\ \hline
  2	& $\coord{14.345}{8.703}$	 & $12.3446$ \\ \hline
\end{tabular}

    \caption{Heuristique $H_0$: Résultats test \no1}
    \label{tab:H0_1}
  \end{table}

  \begin{figure}[H]
    \begin{center}
      \boxed{
      \begin{tikzpicture}[scale=0.7]
        \draw[grided,step=1.0,thin] (-9.000000,-2.000000) grid (7.323262,6.596979);
\draw (-9.000000,0) -- coordinate (x axis mid) (7.323262,0);
\draw (0,-2.000000) -- coordinate (y axis mid) (0,6.596979);
\foreach \x in {-9,...,7}
  \draw (\x,1pt) -- (\x,-3pt) node[anchor=north] {\x};
\foreach \y in {-2,...,6}
  \draw (1pt,\y) -- (-3pt,\y) node[anchor=east] {\y};
\node[interceptor] (I0) at (0.000000,0.000000) {\interceptor};
\node[mobile,anchor=center] (M0) at (1.000000,2.000000) {\mobile};
\node[mobile] at (M0.south east) {$M_0$};
\draw[speed] (M0.center) -- ($ (M0.center) + (-1.000000,0.000000) $);
\node[mobile,anchor=center] (M1) at (0.000000,3.000000) {\mobile};
\node[mobile] at (M1.south east) {$M_1$};
\draw[speed] (M1.center) -- ($ (M1.center) + (0.000000,-1.000000) $);
\node[mobile,anchor=center] (M2) at (2.000000,5.000000) {\mobile};
\node[mobile] at (M2.south east) {$M_2$};
\draw[speed] (M2.center) -- ($ (M2.center) + (1.000000,0.300000) $);
\node[mobile,anchor=center] (M3) at (4.000000,1.000000) {\mobile};
\node[mobile] at (M3.south east) {$M_3$};
\draw[speed] (M3.center) -- ($ (M3.center) + (0.500000,0.500000) $);
\node[mobile,anchor=center] (M4) at (-1.000000,-2.000000) {\mobile};
\node[mobile] at (M4.south east) {$M_4$};
\draw[speed] (M4.center) -- ($ (M4.center) + (-0.500000,0.500000) $);
\draw[camino] (M0) -- (0.000000,2.000000);
\draw[interceptor] (0.000000,0.000000) -- (0.000000,2.000000);
\node[interceptor] at (0.000000,2.000000) {\mobile};
\node[caught] at (M0) {\mobile};
\draw[camino] (M1) -- (0.000000,2.000000);
\draw[interceptor] (0.000000,2.000000) -- (0.000000,2.000000);
\node[interceptor] at (0.000000,2.000000) {\mobile};
\node[caught] at (M1) {\mobile};
\draw[camino] (M2) -- (7.323262,6.596979);
\draw[interceptor] (0.000000,2.000000) -- (7.323262,6.596979);
\node[interceptor] at (7.323262,6.596979) {\mobile};
\node[caught] at (M2) {\mobile};
\draw[camino] (M3) -- (7.248936,4.248936);
\draw[interceptor] (7.323262,6.596979) -- (7.248936,4.248936);
\node[interceptor] at (7.248936,4.248936) {\mobile};
\node[caught] at (M3) {\mobile};
\draw[camino] (M4) -- (-8.089226,5.089226);
\draw[interceptor] (7.248936,4.248936) -- (-8.089226,5.089226);
\node[interceptor] at (-8.089226,5.089226) {\mobile};
\node[caught] at (M4) {\mobile};
\draw[interceptor] (-8.089226,5.089226) node[anchor=south west] {$t=14.178 \text{ u.t.}$};

      \end{tikzpicture}}
    \end{center}
    \caption{Heuristique $H_1$: Test \no1}
    \label{fig:H1_1}
  \end{figure}

  \begin{table}[H]
    \centering
    \begin{tabular}{|c|c|c|}
  \hline\textbf{\No mobile} & \textbf{Position interception} & \textbf{Date interception (u.t.)} \\ \hline 
  0	& $\coord{7.140}{8.876}$	 & $2.5709$ \\ \hline
  1	& $\coord{-1.586}{3.021}$	 & $7.8255$ \\ \hline
  2	& $\coord{-3.278}{3.652}$	 & $8.7281$ \\ \hline
  3	& $\coord{-12.292}{-2.640}$	 & $14.2246$ \\ \hline
  4	& $\coord{-2.281}{6.690}$	 & $21.0671$ \\ \hline
  5	& $\coord{-3.540}{-3.592}$	 & $26.2462$ \\ \hline
\end{tabular}

    \caption{Heuristique $H_1$: Résultats test \no1}
    \label{tab:H1_1}
  \end{table}

  \begin{figure}[H]
    \centering
    \begin{tikzpicture}[yscale=0.5]
      \draw (0,0) -- coordinate (x axis mid) (5,0);
\foreach \x in {0,...,5}
  \draw (\x,1pt) -- (\x,-3pt) node[anchor=north] {\x};
\draw (0,0) -- coordinate (y axis mid) (0,13.000000);
\node[h0] at (1,7.500000) {$H_0$};
\node[h1] at (1,5.500000) {$H_1$};
\foreach \y in {0,1,...,13}
  \draw (1pt,\y) -- (-3pt,\y) node[anchor=east] {\y};
\draw (2.500000,-2) node[anchor=north] {Nombre de mobiles interceptés};
\draw (-0.75,6.500000) node[rotate=90,anchor=south] {Temps nécessaire (u.t)};
\draw[grided,step=1.0,thin] (0,0) grid (5,13.000000);
\node[h0] at (0,0) {\cross};
\draw[h0] (0,0.000000) -- (1,1.000000);
\node[h0] at(1,1.000000) {\cross};
\draw[h0] (1,1.000000) -- (2,1.000000);
\node[h0] at(2,1.000000) {\cross};
\draw[h0] (2,1.000000) -- (3,2.769642);
\node[h0] at(3,2.769642) {\cross};
\draw[h0] (3,2.769642) -- (4,9.018394);
\node[h0] at(4,9.018394) {\cross};
\draw[h0] (4,9.018394) -- (5,12.344612);
\node[h0] at(5,12.344612) {\cross};
\draw[grided,step=1.0,thin] (0,0) grid (5,15.000000);
\node[h1] at (0,0) {\cross};
\draw[h1] (0,0.000000) -- (1,1.000000);
\node[h1] at(1,1.000000) {\cross};
\draw[h1] (1,1.000000) -- (2,1.000000);
\node[h1] at(2,1.000000) {\cross};
\draw[h1] (2,1.000000) -- (3,5.323262);
\node[h1] at(3,5.323262) {\cross};
\draw[h1] (3,5.323262) -- (4,6.497872);
\node[h1] at(4,6.497872) {\cross};
\draw[h1] (4,6.497872) -- (5,14.178453);
\node[h1] at(5,14.178453) {\cross};

    \end{tikzpicture}
    \caption{Comparaison de $H_0$ et de $H_1$: test \no1}
    \label{fig:comp_1}
  \end{figure}

\section{Test \no2: Résultats identiques et mobile non-intercepté}
  \textfile{../tests/test_2/test_2.data}

  \begin{figure}[H]
    \begin{center}
      \boxed{
      \begin{tikzpicture}[scale=1]
        \draw[grided,step=1.0,thin] (-2.000000,-8.000000) grid (14.363306,9.297633);
\draw (-2.000000,0) -- coordinate (x axis mid) (14.363306,0);
\draw (0,-8.000000) -- coordinate (y axis mid) (0,9.297633);
\foreach \x in {-2,...,14}
  \draw (\x,1pt) -- (\x,-3pt) node[anchor=north] {\x};
\foreach \y in {-8,...,9}
  \draw (1pt,\y) -- (-3pt,\y) node[anchor=east] {\y};
\node[interceptor] (I0) at (0.000000,0.000000) {\interceptor};
\node[mobile,anchor=center] (M0) at (2.000000,2.000000) {\mobile};
\node[mobile] at (M0.south east) {$M_0$};
\draw[speed] (M0.center) -- ($ (M0.center) + (-1.000000,0.000000) $);
\node[mobile,anchor=center] (M1) at (-1.000000,-1.000000) {\mobile};
\node[mobile] at (M1.south east) {$M_1$};
\draw[speed] (M1.center) -- ($ (M1.center) + (0.000000,2.000000) $);
\node[mobile,anchor=center] (M2) at (4.000000,5.000000) {\mobile};
\node[mobile] at (M2.south east) {$M_2$};
\draw[speed] (M2.center) -- ($ (M2.center) + (-1.000000,-0.300000) $);
\node[mobile,anchor=center] (M3) at (2.000000,4.000000) {\mobile};
\node[mobile] at (M3.south east) {$M_3$};
\draw[speed] (M3.center) -- ($ (M3.center) + (0.750000,-0.690000) $);
\draw[camino] (M0) -- (0.902832,2.000000);
\draw[interceptor] (0.000000,0.000000) -- (0.902832,2.000000);
\node[interceptor] at (0.902832,2.000000) {\mobile};
\node[caught] at (M0) {\mobile};
\draw[camino] (M2) -- (1.685421,4.305626);
\draw[interceptor] (0.902832,2.000000) -- (1.685421,4.305626);
\node[interceptor] at (1.685421,4.305626) {\mobile};
\node[caught] at (M2) {\mobile};
\draw[camino] (M1) -- (-1.000000,9.297633);
\draw[interceptor] (1.685421,4.305626) -- (-1.000000,9.297633);
\node[interceptor] at (-1.000000,9.297633) {\mobile};
\node[caught] at (M1) {\mobile};
\draw[camino] (M3) -- (14.363306,-7.374242);
\draw[interceptor] (-1.000000,9.297633) -- (14.363306,-7.374242);
\node[interceptor] at (14.363306,-7.374242) {\mobile};
\node[caught] at (M3) {\mobile};
\draw[interceptor] (14.363306,-7.374242) node[anchor=south east] {$t=16.484 \text{ u.t.}$};

      \end{tikzpicture}}
    \end{center}
    \caption{Heuristique $H_0$: Test \no2}
    \label{fig:H0_2}
  \end{figure}

  \begin{table}[H]
    \centering
    \begin{tabular}{|c|c|c|}
  \hline\textbf{\No mobile} & \textbf{Position interception} & \textbf{Date interception (u.t.)} \\ \hline 
  0	& $\coord{0.000}{2.000}$	 & $1.0000$ \\ \hline
  1	& $\coord{0.000}{2.000}$	 & $1.0000$ \\ \hline
  4	& $\coord{-2.385}{-0.615}$	 & $2.7696$ \\ \hline
  3	& $\coord{8.509}{5.509}$	 & $9.0184$ \\ \hline
  2	& $\coord{14.345}{8.703}$	 & $12.3446$ \\ \hline
\end{tabular}

    \caption{Heuristique $H_0$: Résultats test \no2}
    \label{tab:H0_2}
  \end{table}

  \begin{figure}[H]
    \begin{center}
      \boxed{
      \begin{tikzpicture}[scale=1]
        \draw[grided,step=1.0,thin] (-2.000000,-1.000000) grid (5.797277,5.000000);
\draw (-2.000000,0) -- coordinate (x axis mid) (5.797277,0);
\draw (0,-1.000000) -- coordinate (y axis mid) (0,5.000000);
\foreach \x in {-2,...,5}
  \draw (\x,1pt) -- (\x,-3pt) node[anchor=north] {\x};
\foreach \y in {-1,...,5}
  \draw (1pt,\y) -- (-3pt,\y) node[anchor=east] {\y};
\node[interceptor] (I0) at (0.000000,0.000000) {\interceptor};
\node[mobile,anchor=center] (M0) at (2.000000,2.000000) {\mobile};
\node[mobile] at (M0.south east) {$M_0$};
\draw[speed] (M0.center) -- ($ (M0.center) + (-1.500000,0.000000) $);
\node[mobile,anchor=center] (M1) at (-1.000000,-1.000000) {\mobile};
\node[mobile] at (M1.south east) {$M_1$};
\draw[speed] (M1.center) -- ($ (M1.center) + (0.000000,2.000000) $);
\node[mobile,anchor=center] (M2) at (4.000000,5.000000) {\mobile};
\node[mobile] at (M2.south east) {$M_2$};
\draw[speed] (M2.center) -- ($ (M2.center) + (-1.000000,-0.300000) $);
\node[mobile,anchor=center] (M3) at (2.000000,4.000000) {\mobile};
\node[mobile] at (M3.south east) {$M_3$};
\draw[speed] (M3.center) -- ($ (M3.center) + (0.600000,-0.690000) $);
\node[mobile,anchor=center] (M4) at (-2.000000,-1.000000) {\mobile};
\node[mobile] at (M4.south east) {$M_4$};
\draw[speed] (M4.center) -- ($ (M4.center) + (1.000000,6.000000) $);
\draw[camino] (M0) -- (0.460716,2.000000);
\draw[interceptor] (0.000000,0.000000) -- (0.460716,2.000000);
\node[interceptor] at (0.460716,2.000000) {\mobile};
\node[caught] at (M0) {\mobile};
\draw[camino] (M1) -- (-1.000000,2.652004);
\draw[interceptor] (0.460716,2.000000) -- (-1.000000,2.652004);
\node[interceptor] at (-1.000000,2.652004) {\mobile};
\node[caught] at (M1) {\mobile};
\draw[camino] (M2) -- (0.959412,4.087824);
\draw[interceptor] (-1.000000,2.652004) -- (0.959412,4.087824);
\node[interceptor] at (0.959412,4.087824) {\mobile};
\node[caught] at (M2) {\mobile};
\draw[camino] (M3) -- (5.797277,-0.366868);
\draw[interceptor] (0.959412,4.087824) -- (5.797277,-0.366868);
\node[interceptor] at (5.797277,-0.366868) {\mobile};
\node[caught] at (M3) {\mobile};
\draw[interceptor] (5.797277,-0.366868) node[anchor=south east] {$t=6.329 \text{ u.t.}$};

      \end{tikzpicture}}
    \end{center}
    \caption{Heuristique $H_1$: Test \no2}
    \label{fig:H1_2}
  \end{figure}

  \begin{table}[H]
    \centering
    \begin{tabular}{|c|c|c|}
  \hline\textbf{\No mobile} & \textbf{Position interception} & \textbf{Date interception (u.t.)} \\ \hline 
  0	& $\coord{7.140}{8.876}$	 & $2.5709$ \\ \hline
  1	& $\coord{-1.586}{3.021}$	 & $7.8255$ \\ \hline
  2	& $\coord{-3.278}{3.652}$	 & $8.7281$ \\ \hline
  3	& $\coord{-12.292}{-2.640}$	 & $14.2246$ \\ \hline
  4	& $\coord{-2.281}{6.690}$	 & $21.0671$ \\ \hline
  5	& $\coord{-3.540}{-3.592}$	 & $26.2462$ \\ \hline
\end{tabular}

    \caption{Heuristique $H_1$: Résultats test \no2}
    \label{tab:H1_2}
  \end{table}

  \begin{figure}[H]
    \centering
    \begin{tikzpicture}[yscale=0.5]
      \draw (0,0) -- coordinate (x axis mid) (5,0);
\foreach \x in {0,...,5}
  \draw (\x,1pt) -- (\x,-3pt) node[anchor=north] {\x};
\draw (0,0) -- coordinate (y axis mid) (0,7.000000);
\foreach \y in {0,1,...,7}
  \draw (1pt,\y) -- (-3pt,\y) node[anchor=east] {\y};
\draw (2.500000,-2) node[anchor=north] {Nombre de mobiles interceptés};
\draw (-0.75,3.500000) node[rotate=90,anchor=south] {Temps nécessaire (u.t)};
\draw[grided,step=1.0,thin] (0,0) grid (5,7.000000);
\node[h0] at (0,0) {\cross};
\draw[h0] (0,0.000000) -- (1,1.026189);
\node[h0] at(1,1.026189) {\cross};
\draw[h0] (1,1.026189) -- (2,1.826002);
\node[h0] at(2,1.826002) {\cross};
\draw[h0] (2,1.826002) -- (3,3.040588);
\node[h0] at(3,3.040588) {\cross};
\draw[h0] (3,3.040588) -- (4,6.328794);
\node[h0] at(4,6.328794) {\cross};
\draw[grided,step=1.0,thin] (0,0) grid (5,7.000000);
\node[h1] at (0,0) {\cross};
\draw[h1] (0,0.000000) -- (1,1.026189);
\node[h1] at(1,1.026189) {\cross};
\draw[h1] (1,1.026189) -- (2,1.826002);
\node[h1] at(2,1.826002) {\cross};
\draw[h1] (2,1.826002) -- (3,3.040588);
\node[h1] at(3,3.040588) {\cross};
\draw[h1] (3,3.040588) -- (4,6.328794);
\node[h1] at(4,6.328794) {\cross};

    \end{tikzpicture}
    \caption{Comparaison de $H_0$ et de $H_1$: test \no2}
    \label{fig:comp_2}
  \end{figure}


\section{Test \no3: Mobiles positionnés aléatoirement}
  \textfile{../tests/test_3/test_3.data}

  \begin{figure}[H]
    \begin{center}
      \boxed{
      \begin{tikzpicture}[scale=0.5]
        \draw[grided,step=1.0,thin] (-15.000000,-9.000000) grid (8.360891,9.000000);
\draw (-15.000000,0) -- coordinate (x axis mid) (8.360891,0);
\draw (0,-9.000000) -- coordinate (y axis mid) (0,9.000000);
\foreach \x in {-15,...,8}
  \draw (\x,1pt) -- (\x,-3pt) node[anchor=north] {\x};
\foreach \y in {-9,...,9}
  \draw (1pt,\y) -- (-3pt,\y) node[anchor=east] {\y};
\node[interceptor] (I0) at (2.000000,9.000000) {\interceptor};
\node[mobile,anchor=center] (M0) at (7.140297,8.876400) {\mobile};
\node[mobile] at (M0.south east) {$M_0$};
\node[mobile,anchor=center] (M1) at (4.767767,5.291864) {\mobile};
\node[mobile] at (M1.south east) {$M_1$};
\draw[speed] (M1.center) -- ($ (M1.center) + (-0.811979,-0.290230) $);
\node[mobile,anchor=center] (M2) at (-7.277143,-4.215198) {\mobile};
\node[mobile] at (M2.south east) {$M_2$};
\draw[speed] (M2.center) -- ($ (M2.center) + (0.458222,0.901331) $);
\node[mobile,anchor=center] (M3) at (-9.317039,-8.399123) {\mobile};
\node[mobile] at (M3.south east) {$M_3$};
\draw[speed] (M3.center) -- ($ (M3.center) + (-0.209148,0.404874) $);
\node[mobile,anchor=center] (M4) at (8.360891,5.869070) {\mobile};
\node[mobile] at (M4.south east) {$M_4$};
\draw[speed] (M4.center) -- ($ (M4.center) + (-0.505128,0.038969) $);
\node[mobile,anchor=center] (M5) at (2.371453,-8.866265) {\mobile};
\node[mobile] at (M5.south east) {$M_5$};
\draw[speed] (M5.center) -- ($ (M5.center) + (-0.225212,0.200970) $);
\draw[camino] (M1) -- (2.957371,4.644764);
\draw[interceptor] (2.000000,9.000000) -- (2.957371,4.644764);
\node[interceptor] at (2.957371,4.644764) {\mobile};
\node[caught] at (M1) {\mobile};
\draw[camino] (M4) -- (6.317120,6.026740);
\draw[interceptor] (2.957371,4.644764) -- (6.317120,6.026740);
\node[interceptor] at (6.317120,6.026740) {\mobile};
\node[caught] at (M4) {\mobile};
\draw[interceptor] (6.317120,6.026740) -- (7.140297,8.876400);
\node[interceptor] at (7.140297,8.876400) {\mobile};
\node[caught] at (M0) {\mobile};
\draw[camino] (M2) -- (-2.411295,5.356014);
\draw[interceptor] (7.140297,8.876400) -- (-2.411295,5.356014);
\node[interceptor] at (-2.411295,5.356014) {\mobile};
\node[caught] at (M2) {\mobile};
\draw[camino] (M5) -- (-1.263106,-5.622933);
\draw[interceptor] (-2.411295,5.356014) -- (-1.263106,-5.622933);
\node[interceptor] at (-1.263106,-5.622933) {\mobile};
\node[caught] at (M5) {\mobile};
\draw[camino] (M3) -- (-14.218112,1.088498);
\draw[interceptor] (-1.263106,-5.622933) -- (-14.218112,1.088498);
\node[interceptor] at (-14.218112,1.088498) {\mobile};
\node[caught] at (M3) {\mobile};
\draw[interceptor] (-14.218112,1.088498) node[anchor=south west] {$t=23.434 \text{ u.t.}$};

      \end{tikzpicture}}
    \end{center}
    \caption{Heuristique $H_0$: Test \no3}
    \label{fig:H0_3}
  \end{figure}

  \begin{table}[H]
    \centering
    \begin{tabular}{|c|c|c|}
  \hline\textbf{\No mobile} & \textbf{Position interception} & \textbf{Date interception (u.t.)} \\ \hline 
  0	& $\coord{0.000}{2.000}$	 & $1.0000$ \\ \hline
  1	& $\coord{0.000}{2.000}$	 & $1.0000$ \\ \hline
  4	& $\coord{-2.385}{-0.615}$	 & $2.7696$ \\ \hline
  3	& $\coord{8.509}{5.509}$	 & $9.0184$ \\ \hline
  2	& $\coord{14.345}{8.703}$	 & $12.3446$ \\ \hline
\end{tabular}

    \caption{Heuristique $H_0$: Résultats test \no3}
    \label{tab:H0_3}
  \end{table}

  \begin{figure}[H]
    \begin{center}
      \boxed{
      \begin{tikzpicture}[scale=0.5]
        \draw[grided,step=1.0,thin] (-13.000000,-9.000000) grid (8.360891,9.000000);
\draw (-13.000000,0) -- coordinate (x axis mid) (8.360891,0);
\draw (0,-9.000000) -- coordinate (y axis mid) (0,9.000000);
\foreach \x in {-13,...,8}
  \draw (\x,1pt) -- (\x,-3pt) node[anchor=north] {\x};
\foreach \y in {-9,...,9}
  \draw (1pt,\y) -- (-3pt,\y) node[anchor=east] {\y};
\node[interceptor] (I0) at (2.000000,9.000000) {\interceptor};
\node[mobile,anchor=center] (M0) at (7.140297,8.876400) {\mobile};
\node[mobile] at (M0.south east) {$M_0$};
\node[mobile,anchor=center] (M1) at (4.767767,5.291864) {\mobile};
\node[mobile] at (M1.south east) {$M_1$};
\draw[speed] (M1.center) -- ($ (M1.center) + (-0.811979,-0.290230) $);
\node[mobile,anchor=center] (M2) at (-7.277143,-4.215198) {\mobile};
\node[mobile] at (M2.south east) {$M_2$};
\draw[speed] (M2.center) -- ($ (M2.center) + (0.458222,0.901331) $);
\node[mobile,anchor=center] (M3) at (-9.317039,-8.399123) {\mobile};
\node[mobile] at (M3.south east) {$M_3$};
\draw[speed] (M3.center) -- ($ (M3.center) + (-0.209148,0.404874) $);
\node[mobile,anchor=center] (M4) at (8.360891,5.869070) {\mobile};
\node[mobile] at (M4.south east) {$M_4$};
\draw[speed] (M4.center) -- ($ (M4.center) + (-0.505128,0.038969) $);
\node[mobile,anchor=center] (M5) at (2.371453,-8.866265) {\mobile};
\node[mobile] at (M5.south east) {$M_5$};
\draw[speed] (M5.center) -- ($ (M5.center) + (-0.225212,0.200970) $);
\draw[interceptor] (2.000000,9.000000) -- (7.140297,8.876400);
\node[interceptor] at (7.140297,8.876400) {\mobile};
\node[caught] at (M0) {\mobile};
\draw[camino] (M1) -- (-1.586399,3.020660);
\draw[interceptor] (7.140297,8.876400) -- (-1.586399,3.020660);
\node[interceptor] at (-1.586399,3.020660) {\mobile};
\node[caught] at (M1) {\mobile};
\draw[camino] (M2) -- (-3.277717,3.651746);
\draw[interceptor] (-1.586399,3.020660) -- (-3.277717,3.651746);
\node[interceptor] at (-3.277717,3.651746) {\mobile};
\node[caught] at (M2) {\mobile};
\draw[camino] (M3) -- (-12.292086,-2.639952);
\draw[interceptor] (-3.277717,3.651746) -- (-12.292086,-2.639952);
\node[interceptor] at (-12.292086,-2.639952) {\mobile};
\node[caught] at (M3) {\mobile};
\draw[camino] (M4) -- (-2.280671,6.690032);
\draw[interceptor] (-12.292086,-2.639952) -- (-2.280671,6.690032);
\node[interceptor] at (-2.280671,6.690032) {\mobile};
\node[caught] at (M4) {\mobile};
\draw[camino] (M5) -- (-3.539516,-3.591557);
\draw[interceptor] (-2.280671,6.690032) -- (-3.539516,-3.591557);
\node[interceptor] at (-3.539516,-3.591557) {\mobile};
\node[caught] at (M5) {\mobile};
\draw[interceptor] (-3.539516,-3.591557) node[anchor=south east] {$t=26.246 \text{ u.t.}$};

      \end{tikzpicture}}
    \end{center}
    \caption{Heuristique $H_1$: Test \no3}
    \label{fig:H1_3}
  \end{figure}

  \begin{table}[H]
    \centering
    \begin{tabular}{|c|c|c|}
  \hline\textbf{\No mobile} & \textbf{Position interception} & \textbf{Date interception (u.t.)} \\ \hline 
  0	& $\coord{7.140}{8.876}$	 & $2.5709$ \\ \hline
  1	& $\coord{-1.586}{3.021}$	 & $7.8255$ \\ \hline
  2	& $\coord{-3.278}{3.652}$	 & $8.7281$ \\ \hline
  3	& $\coord{-12.292}{-2.640}$	 & $14.2246$ \\ \hline
  4	& $\coord{-2.281}{6.690}$	 & $21.0671$ \\ \hline
  5	& $\coord{-3.540}{-3.592}$	 & $26.2462$ \\ \hline
\end{tabular}

    \caption{Heuristique $H_1$: Résultats test \no3}
    \label{tab:H1_3}
  \end{table}

  \begin{figure}[H]
    \centering
    \begin{tikzpicture}[yscale=0.2]
      \draw (0,0) -- coordinate (x axis mid) (6,0);
\foreach \x in {0,...,6}
  \draw (\x,1pt) -- (\x,-3pt) node[anchor=north] {\x};
\draw (0,0) -- coordinate (y axis mid) (0,24.000000);
\node[h0] at (1,13.000000) {$H_0$};
\node[h1] at (1,11.000000) {$H_1$};
\foreach \y in {0,2,...,24}
  \draw (1pt,\y) -- (-3pt,\y) node[anchor=east] {\y};
\draw (3.000000,-2) node[anchor=north] {Nombre de mobiles interceptés};
\draw (-0.75,12.000000) node[rotate=90,anchor=south] {Temps nécessaire (u.t)};
\draw[grided,step=1.0,thin] (0,0) grid (6,24.000000);
\node[h0] at (0,0) {\cross};
\draw[h0] (0,0.000000) -- (1,2.229610);
\node[h0] at(1,2.229610) {\cross};
\draw[h0] (1,2.229610) -- (2,4.046046);
\node[h0] at(2,4.046046) {\cross};
\draw[h0] (2,4.046046) -- (3,5.529133);
\node[h0] at(3,5.529133) {\cross};
\draw[h0] (3,5.529133) -- (4,10.618976);
\node[h0] at(4,10.618976) {\cross};
\draw[h0] (4,10.618976) -- (5,16.138388);
\node[h0] at(5,16.138388) {\cross};
\draw[h0] (5,16.138388) -- (6,23.433514);
\node[h0] at(6,23.433514) {\cross};
\draw[grided,step=1.0,thin] (0,0) grid (6,27.000000);
\node[h1] at (0,0) {\cross};
\draw[h1] (0,0.000000) -- (1,2.570891);
\node[h1] at(1,2.570891) {\cross};
\draw[h1] (1,2.570891) -- (2,7.825530);
\node[h1] at(2,7.825530) {\cross};
\draw[h1] (2,7.825530) -- (3,8.728141);
\node[h1] at(3,8.728141) {\cross};
\draw[h1] (3,8.728141) -- (4,14.224601);
\node[h1] at(4,14.224601) {\cross};
\draw[h1] (4,14.224601) -- (5,21.067061);
\node[h1] at(5,21.067061) {\cross};
\draw[h1] (5,21.067061) -- (6,26.246245);
\node[h1] at(6,26.246245) {\cross};

    \end{tikzpicture}
    \caption{Comparaison de $H_0$ et de $H_1$: test \no3}
    \label{fig:comp_3}
  \end{figure}

  \section{Test \no4: Heuristique $H_1$ plus rapide}
  \textfile{../tests/test_4/test_4.data}

  \begin{figure}[H]
    \begin{center}
      \boxed{
      \begin{tikzpicture}[scale=0.5]
        \draw[grided,step=1.0,thin] (-2.000000,-8.000000) grid (14.363306,9.297633);
\draw (-2.000000,0) -- coordinate (x axis mid) (14.363306,0);
\draw (0,-8.000000) -- coordinate (y axis mid) (0,9.297633);
\foreach \x in {-2,...,14}
  \draw (\x,1pt) -- (\x,-3pt) node[anchor=north] {\x};
\foreach \y in {-8,...,9}
  \draw (1pt,\y) -- (-3pt,\y) node[anchor=east] {\y};
\node[interceptor] (I0) at (0.000000,0.000000) {\interceptor};
\node[mobile,anchor=center] (M0) at (2.000000,2.000000) {\mobile};
\node[mobile] at (M0.south east) {$M_0$};
\draw[speed] (M0.center) -- ($ (M0.center) + (-1.000000,0.000000) $);
\node[mobile,anchor=center] (M1) at (-1.000000,-1.000000) {\mobile};
\node[mobile] at (M1.south east) {$M_1$};
\draw[speed] (M1.center) -- ($ (M1.center) + (0.000000,2.000000) $);
\node[mobile,anchor=center] (M2) at (4.000000,5.000000) {\mobile};
\node[mobile] at (M2.south east) {$M_2$};
\draw[speed] (M2.center) -- ($ (M2.center) + (-1.000000,-0.300000) $);
\node[mobile,anchor=center] (M3) at (2.000000,4.000000) {\mobile};
\node[mobile] at (M3.south east) {$M_3$};
\draw[speed] (M3.center) -- ($ (M3.center) + (0.750000,-0.690000) $);
\draw[camino] (M0) -- (0.902832,2.000000);
\draw[interceptor] (0.000000,0.000000) -- (0.902832,2.000000);
\node[interceptor] at (0.902832,2.000000) {\mobile};
\node[caught] at (M0) {\mobile};
\draw[camino] (M2) -- (1.685421,4.305626);
\draw[interceptor] (0.902832,2.000000) -- (1.685421,4.305626);
\node[interceptor] at (1.685421,4.305626) {\mobile};
\node[caught] at (M2) {\mobile};
\draw[camino] (M1) -- (-1.000000,9.297633);
\draw[interceptor] (1.685421,4.305626) -- (-1.000000,9.297633);
\node[interceptor] at (-1.000000,9.297633) {\mobile};
\node[caught] at (M1) {\mobile};
\draw[camino] (M3) -- (14.363306,-7.374242);
\draw[interceptor] (-1.000000,9.297633) -- (14.363306,-7.374242);
\node[interceptor] at (14.363306,-7.374242) {\mobile};
\node[caught] at (M3) {\mobile};
\draw[interceptor] (14.363306,-7.374242) node[anchor=south east] {$t=16.484 \text{ u.t.}$};

      \end{tikzpicture}}
    \end{center}
    \caption{Heuristique $H_0$: Test \no4}
    \label{fig:H0_4}
  \end{figure}

  \begin{table}[H]
    \centering
    \begin{tabular}{|c|c|c|}
  \hline\textbf{\No mobile} & \textbf{Position interception} & \textbf{Date interception (u.t.)} \\ \hline 
  0	& $\coord{0.000}{2.000}$	 & $1.0000$ \\ \hline
  1	& $\coord{0.000}{2.000}$	 & $1.0000$ \\ \hline
  4	& $\coord{-2.385}{-0.615}$	 & $2.7696$ \\ \hline
  3	& $\coord{8.509}{5.509}$	 & $9.0184$ \\ \hline
  2	& $\coord{14.345}{8.703}$	 & $12.3446$ \\ \hline
\end{tabular}

    \caption{Heuristique $H_0$: Résultats test \no4}
    \label{tab:H0_4}
  \end{table}

  \begin{figure}[H]
    \begin{center}
      \boxed{
      \begin{tikzpicture}[scale=1]
        \draw[grided,step=1.0,thin] (-1.000000,-2.000000) grid (7.784527,5.000000);
\draw (-1.000000,0) -- coordinate (x axis mid) (7.784527,0);
\draw (0,-2.000000) -- coordinate (y axis mid) (0,5.000000);
\foreach \x in {-1,...,7}
  \draw (\x,1pt) -- (\x,-3pt) node[anchor=north] {\x};
\foreach \y in {-2,...,5}
  \draw (1pt,\y) -- (-3pt,\y) node[anchor=east] {\y};
\node[interceptor] (I0) at (0.000000,0.000000) {\interceptor};
\node[mobile,anchor=center] (M0) at (2.000000,2.000000) {\mobile};
\node[mobile] at (M0.south east) {$M_0$};
\draw[speed] (M0.center) -- ($ (M0.center) + (-1.000000,0.000000) $);
\node[mobile,anchor=center] (M1) at (-1.000000,-1.000000) {\mobile};
\node[mobile] at (M1.south east) {$M_1$};
\draw[speed] (M1.center) -- ($ (M1.center) + (0.000000,2.000000) $);
\node[mobile,anchor=center] (M2) at (4.000000,5.000000) {\mobile};
\node[mobile] at (M2.south east) {$M_2$};
\draw[speed] (M2.center) -- ($ (M2.center) + (-1.000000,-0.300000) $);
\node[mobile,anchor=center] (M3) at (2.000000,4.000000) {\mobile};
\node[mobile] at (M3.south east) {$M_3$};
\draw[speed] (M3.center) -- ($ (M3.center) + (0.750000,-0.690000) $);
\draw[camino] (M0) -- (0.902832,2.000000);
\draw[interceptor] (0.000000,0.000000) -- (0.902832,2.000000);
\node[interceptor] at (0.902832,2.000000) {\mobile};
\node[caught] at (M0) {\mobile};
\draw[camino] (M1) -- (-1.000000,3.844238);
\draw[interceptor] (0.902832,2.000000) -- (-1.000000,3.844238);
\node[interceptor] at (-1.000000,3.844238) {\mobile};
\node[caught] at (M1) {\mobile};
\draw[camino] (M2) -- (0.715771,4.014731);
\draw[interceptor] (-1.000000,3.844238) -- (0.715771,4.014731);
\node[interceptor] at (0.715771,4.014731) {\mobile};
\node[caught] at (M2) {\mobile};
\draw[camino] (M3) -- (7.784527,-1.321765);
\draw[interceptor] (0.715771,4.014731) -- (7.784527,-1.321765);
\node[interceptor] at (7.784527,-1.321765) {\mobile};
\node[caught] at (M3) {\mobile};
\draw[interceptor] (7.784527,-1.321765) node[anchor=south east] {$t=7.713 \text{ u.t.}$};

      \end{tikzpicture}}
    \end{center}
    \caption{Heuristique $H_1$: Test \no4}
    \label{fig:H1_4}
  \end{figure}

  \begin{table}[H]
    \centering
    \begin{tabular}{|c|c|c|}
  \hline\textbf{\No mobile} & \textbf{Position interception} & \textbf{Date interception (u.t.)} \\ \hline 
  0	& $\coord{0.903}{2.000}$	 & $1.0972$ \\ \hline
  1	& $\coord{-1.000}{3.844}$	 & $2.4221$ \\ \hline
  2	& $\coord{0.716}{4.015}$	 & $3.2842$ \\ \hline
  3	& $\coord{7.785}{-1.322}$	 & $7.7127$ \\ \hline
\end{tabular}

    \caption{Heuristique $H_1$: Résultats test \no4}
    \label{tab:H1_4}
  \end{table}

  \begin{figure}[H]
    \centering
    \begin{tikzpicture}[yscale=0.35]
      \draw (0,0) -- coordinate (x axis mid) (4,0);
\foreach \x in {0,...,4}
  \draw (\x,1pt) -- (\x,-3pt) node[anchor=north] {\x};
\draw (0,0) -- coordinate (y axis mid) (0,17.000000);
\node[h0] at (1,9.500000) {$H_0$};
\node[h1] at (1,7.500000) {$H_1$};
\foreach \y in {0,2,...,17}
  \draw (1pt,\y) -- (-3pt,\y) node[anchor=east] {\y};
\draw (2.000000,-2) node[anchor=north] {Nombre de mobiles interceptés};
\draw (-0.75,8.500000) node[rotate=90,anchor=south] {Temps nécessaire (u.t)};
\draw[grided,step=1.0,thin] (0,0) grid (4,17.000000);
\node[h0] at (0,0) {\cross};
\draw[h0] (0,0.000000) -- (1,1.097168);
\node[h0] at(1,1.097168) {\cross};
\draw[h0] (1,1.097168) -- (2,2.314579);
\node[h0] at(2,2.314579) {\cross};
\draw[h0] (2,2.314579) -- (3,5.148816);
\node[h0] at(3,5.148816) {\cross};
\draw[h0] (3,5.148816) -- (4,16.484408);
\node[h0] at(4,16.484408) {\cross};
\draw[grided,step=1.0,thin] (0,0) grid (4,8.000000);
\node[h1] at (0,0) {\cross};
\draw[h1] (0,0.000000) -- (1,1.097168);
\node[h1] at(1,1.097168) {\cross};
\draw[h1] (1,1.097168) -- (2,2.422119);
\node[h1] at(2,2.422119) {\cross};
\draw[h1] (2,2.422119) -- (3,3.284229);
\node[h1] at(3,3.284229) {\cross};
\draw[h1] (3,3.284229) -- (4,7.712703);
\node[h1] at(4,7.712703) {\cross};

    \end{tikzpicture}
    \caption{Comparaison de $H_0$ et de $H_1$: test \no4}
    \label{fig:comp_4}
  \end{figure}

  
  \chapter{Conclusion}
    Le but de ce projet était de maximiser le nombre de mobiles interceptés en minimisant le temps d'interception.
Nous avons donc construit deux heuristiques capable de fournir une solution stable et d'interpréter les résultats.
On remarque que les heuristiques tendent vers des temps d'interceptions similaires lorsqu'on essaye d'intercepter tous les mobiles mais que pour un nombre faible de mobile, l'heuristique $H_0$ semble être plus efficace.
Cependant, elle est plus couteuse en temps CPU que l'heuristique $H_1$ puisqu'elle demande de recalculer le temps d'interception minimum à chaque itération.
Enfin ces deux solutions ne fournissent pas la solution optimale et donc, il est possible de définir des variantes qui pourraient obtenir de meilleurs résultats.
\section*{Propositions d'améliorations}

On pourrait considérer que l'intercepteur se déplace à une vitesse variable (majorée par une vitesse max $v_1$) et chercher à trouver la vitesse $v$ qui nous permettrait d'intercepter un mobile plus rapidement.

  
  \section*{Remerciements}
    Nous tenons à remercier Christophe DUHAMEL qui nous a accompagné dans tout le cheminement de notre projet et qui nous a conseillé quant aux méthodes à employer pour réaliser nos heuristiques.

    Egalement, nous souhaitons remercier Luc MARCHAND pour son aide dans la détermination du calcul nécessaire à l'obtention de l'angle $\alpha$ permettant d'orienter l'intercepteur, ainsi que Gilles LEBORGNE qui nous a aidé à déterminer si la solution était facilement résolvable.

  \section*{Outils utilisés}
    Pour la rédaction de ce rapport, nous avons utilisé \LaTeX{} avec le module \TikZ{} pour les graphiques, et \texttt{minted} pour l'affichage des fichiers.

    Le code correspondant aux graphiques est généré par le programme et inclus à la compilation du rapport.

    Notre code source est écrit en langage \texttt{C} et est disponible sur la plateforme GitHub à l'adresse :
    \begin{center}
      \href{https://github.com/theCmaker/Projet\_ZZ1}{https://github.com/theCmaker/Projet\_ZZ1}
    \end{center}

  \listoffigures
  \listoftables
  \begingroup
    \let\clearpage\relax
    \listoflistings
  \endgroup
\end{document}
