\section{Calcul d'interception}

On modélise le déplacement de l'intercepteur par la fonction suivante : 
\[
i(t,\alpha) = 
\left(
\begin{array}{r c l}
 x_1 + t*v_1*cos(\alpha)\\
 y_1 + t*v_1*sin(\alpha)
\end{array}
\right)
\]
avec $t \in \Re^+$ et $\alpha \in [-\pi,\pi]$.

On modélise de même la même manière le déplacement du mobile :
\[
m(t) = 
\left(
\begin{array}{r c l}
 x_0 + t*v^{x}_0\\
 y_0 + t*v^{y}_1
\end{array}
\right)
\]
avec $t \in \Re^+$.

On doit donc résoudre l'équation suivante afin de calculer le temps d'interception d'un mobile:

\[
\left\{
\begin{array}{r c l}
x_1 + t*v_1*cos(\alpha) &=& x_0 + t*v^{x}_0\\
y_1 + t*v_1*sin(\alpha) &=& y_0 + t*v^{y}_1
\end{array}
\right.
\]

La valeur est donnée par la résolution de l'équation $a*cos(\alpha)+b*sin(\alpha) = c$ avec:
\[
\left\{
\begin{array}{r c l}
a &=& y_0 - y_1\\
b &=& x_1 - x_0\\
c &=& \frac{a*v^{x}_0 +b*v^{y}_1}{v_1}
\end{array}
\right.
\]

On obtient alors 2 possibilités pour la date t : 
$\frac{-b}{-v^{x}_0 + v_1*cos(\alpha)}$ et
$\frac{a}{-v^{y}_0 + v_1*sin(\alpha)}$

La fonction d'interception teste alors les positions obtenues avec les deux dates et retient celle qui fonctionne et qui est minimale.

\section{Heuristique $H_0$}

\section{Heuristique $H_1$}
