% Chained List
% Author: Pierre-Loup Pissavy
\usepackage{pgf,tikz}
\usetikzlibrary{calc,arrows}
\usepackage{amsmath}

\makeatletter

% Chain shape
\pgfdeclareshape{chain2}{
  % The 'minimum width' and 'minimum height' keys, not the content, determine
  % the size
  \savedanchor\northeast{%
    \pgfmathsetlength\pgf@x{\pgfshapeminwidth}%
    \pgfmathsetlength\pgf@y{\pgfshapeminheight}%
    \pgf@x=0.5\pgf@x
    \pgf@y=0.5\pgf@y
  }
  % This is redundant, but makes some things easier:
  \savedanchor\southwest{%
    \pgfmathsetlength\pgf@x{\pgfshapeminwidth}%
    \pgfmathsetlength\pgf@y{\pgfshapeminheight}%
    \pgf@x=-0.5\pgf@x
    \pgf@y=-0.5\pgf@y
  }
  % Inherit from rectangle
  \inheritanchorborder[from=rectangle]

  % Define same anchor a normal rectangle has
  \anchor{center}{\pgfpointorigin}
  \anchor{north}{\northeast \pgf@x=-.66\pgf@x}
  \anchor{east}{\northeast \pgf@y=0pt}
  \anchor{south}{\southwest \pgf@x=.5\pgf@x}
  \anchor{west}{\southwest \pgf@y=0pt}
  \anchor{north east}{\northeast}
  \anchor{north west}{\northeast \pgf@x=-\pgf@x}
  \anchor{south west}{\southwest}
  \anchor{south east}{\southwest \pgf@x=-\pgf@x}

  % Define anchors

  \anchor{text}{
    \pgf@process{\southwest}%
    \pgf@x=.66\pgf@x%
    \pgf@y=0\pgf@y%
    \advance\pgf@x by -0.5\wd\pgfnodeparttextbox%
    \advance\pgf@y by -0.5\ht\pgfnodeparttextbox%
    \advance\pgf@y by +.5\dp\pgfnodeparttextbox%
  }
  \anchor{ptr1}{
    \pgf@process{\pgfpointorigin}%
    \advance\pgf@x by -0.5\wd\pgfnodeparttextbox%
    \advance\pgf@y by -0.5\ht\pgfnodeparttextbox%
    \advance\pgf@y by +.5\dp\pgfnodeparttextbox%
  }
  \anchor{ptr2}{
    \pgf@process{\northeast}%
    \pgf@x=.66\pgf@x%
    \pgf@y=0\pgf@y%
    \advance\pgf@x by -0.5\wd\pgfnodeparttextbox%
    \advance\pgf@y by -0.5\ht\pgfnodeparttextbox%
    \advance\pgf@y by +.5\dp\pgfnodeparttextbox%
  }
  \anchor{next}{
    \pgf@process{\northeast}%
    \pgf@x=0.66\pgf@x%
    \pgf@y=0\pgf@y%
  }
  % Draw the rectangle box
  \backgroundpath{
    % Rectangle box
    \pgfpathrectanglecorners{\southwest}{\northeast}
    \pgfpathrectanglecorners{\southwest \pgf@x=0.33\pgf@x}{\northeast \pgf@x=0.33\pgf@x}%Separator
    \pgfmathsetlength\pgf@x{1.6ex} % size depends
    \pgfclosepath
  }
}
% Ptr shape
\pgfdeclareshape{ptr}{
  % The 'minimum width' and 'minimum height' keys, not the content, determine
  % the size
  \savedanchor\northeast{%
    \pgfmathsetlength\pgf@x{\pgfshapeminwidth}%
    \pgfmathsetlength\pgf@y{\pgfshapeminheight}%
    \pgf@x=0.5\pgf@x
    \pgf@y=0.5\pgf@y
  }
  % This is redundant, but makes some things easier:
  \savedanchor\southwest{%
    \pgfmathsetlength\pgf@x{\pgfshapeminwidth}%
    \pgfmathsetlength\pgf@y{\pgfshapeminheight}%
    \pgf@x=-0.5\pgf@x
    \pgf@y=-0.5\pgf@y
  }
  % Inherit from rectangle
  \inheritanchorborder[from=rectangle]

  % Define same anchor a normal rectangle has
  \anchor{center}{\pgfpointorigin}
  \anchor{north}{\northeast \pgf@x=0pt}
  \anchor{east}{\northeast \pgf@y=0pt}
  \anchor{south}{\southwest \pgf@x=0pt}
  \anchor{west}{\southwest \pgf@y=0pt}
  \anchor{north east}{\northeast}
  \anchor{north west}{\northeast \pgf@x=-\pgf@x}
  \anchor{south west}{\southwest}
  \anchor{south east}{\southwest \pgf@x=-\pgf@x}

  % Define anchors
  \anchor{text}{
    \pgf@process{\pgfpointorigin}%
    \pgf@x=.5\pgf@x%
    \pgf@y=0\pgf@y%
    \advance\pgf@x by -0.5\wd\pgfnodeparttextbox%
    \advance\pgf@y by -0.5\ht\pgfnodeparttextbox%
    \advance\pgf@y by +.5\dp\pgfnodeparttextbox%
  }
  \anchor{next}{\pgf@process{\pgfpointorigin}}

  % Draw the rectangle box
  \backgroundpath{
    % Rectangle box
    \pgfpathrectanglecorners{\southwest}{\northeast}
    \pgfmathsetlength\pgf@x{1.6ex} % size depends
    \pgfclosepath
  }
}

\tikzset{every chain2 node/.style={draw,minimum width=2.1cm,minimum height=0.7cm}}
\tikzset{every ptr node/.style={draw,minimum width=0.7cm,minimum height=0.7cm}}

\makeatother